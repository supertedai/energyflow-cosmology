\documentclass{article}
\usepackage{amsmath, amssymb}
\usepackage{graphicx}
\usepackage{hyperref}

\title{Variable Light Speed and the s\textsubscript{0} / s\textsubscript{1} Entropic States}
\author{Morten Magnusson}
\date{\today}

\begin{document}
\maketitle

\begin{abstract}
This paper explores the relationship between the speed of light and the two fundamental entropic states in Energy-Flow Cosmology (EFC): the ground state s\textsubscript{0}, and the active-flow state s\textsubscript{1}. 
We outline how light-speed is not a universal constant, but an emergent limit of information transfer set by the entropic gradient between these states. 
The transition from s\textsubscript{0} to s\textsubscript{1} defines the available degrees of freedom for energy, information, and curvature. 
This provides a thermodynamic interpretation of light-speed variability across different cosmic environments without invoking exotic physics.
\end{abstract}

\section{Introduction}

In the EFC framework, the speed of light is not a fixed ontological quantity, but the consequence of underlying thermodynamic structure. 
The fundamental entropic states, s\textsubscript{0} and s\textsubscript{1}, describe two regimes:

\begin{itemize}
\item s\textsubscript{0}: minimal entropy, low available degrees of freedom, near-ground energy configuration.
\item s\textsubscript{1}: active-flow regime, increased entropy, structure formation, causal gradients.
\end{itemize}

The transition between these states defines the local information capacity and therefore the upper bound on propagation velocity. 
Thus, light-speed emerges as a constraint of the thermodynamic environment, rather than an independent constant.

\section{The s\textsubscript{0} State: Low-Entropy Ground Configuration}

The s\textsubscript{0} state corresponds to a high-coherence, low-entropy regime where:

\begin{itemize}
\item curvature is minimal,
\item the energy-flow field is weak,
\item information capacity is low,
\item and degrees of freedom are constrained.
\end{itemize}

In this state, the effective speed of light approaches a maximum because the energy landscape is flat and minimally resistive. 
There is little entropic drag, and propagation is limited primarily by the intrinsic properties of the local energy grid.

\section{The s\textsubscript{1} State: Active Entropic Flow}

In contrast, the s\textsubscript{1} state is characterized by:

\begin{itemize}
\item increased entropy,
\item active energy flow,
\item structure formation,
\item curvature and gradients,
\item local increases in information density.
\end{itemize}

In this regime, the effective speed of light can decrease because information transfer experiences entropic resistance. 
The gradient between s\textsubscript{0} and s\textsubscript{1} defines the dynamic “capacity field” that shapes light-speed locally.

\section{Variable Light Speed as an Entropic Phenomenon}

Light-speed variability arises from differences in the s\textsubscript{0}/s\textsubscript{1} balance:

\begin{equation}
c_{\mathrm{eff}} = c_0 \, f(s_0, s_1)
\end{equation}

where $c_{\mathrm{eff}}$ is the local effective light-speed, and $f(s_0, s_1)$ encodes the entropic structure of the region.

High s\textsubscript{1} regions—clusters, halos, active flow zones—exhibit reduced effective light-speed.  
Low s\textsubscript{1}, high s\textsubscript{0} regions—voids, underdense space—allow higher propagation limits.

This interpretation removes the need for exotic fields or variable-c modifications. 
Instead, light-speed becomes a natural emergent property of the thermodynamic landscape.

\section{Implications for Cosmology}

This model predicts:

\begin{itemize}
\item Light-speed anisotropy across large-scale structure.
\item Redshift deviations without invoking expansion-based metrics.
\item Natural alternatives to dark energy interpretation.
\item Modified time-delay behaviour in lensing environments.
\item Predictable light-speed variation across entropic gradients.
\end{itemize}

These effects follow directly from the s\textsubscript{0}/s\textsubscript{1} distribution of the cosmic grid.

\section{Conclusion}

The speed of light emerges from the entropic structure of spacetime. 
The s\textsubscript{0}/s\textsubscript{1} states define the local energy-flow environment, which in turn constrains the maximum rate of information propagation. 
This framework provides a unified, thermodynamic interpretation of variable light-speed consistent with the broader Energy-Flow Cosmology model.

\end{document}

