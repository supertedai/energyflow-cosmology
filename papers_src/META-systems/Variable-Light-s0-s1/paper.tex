\documentclass[12pt]{article}

\usepackage[a4paper, margin=2.5cm]{geometry}
\usepackage{amsmath, amssymb}
\usepackage{hyperref}

\title{Variable Light Speed and the s\textsubscript{0} / s\textsubscript{1} Entropic States}
\author{Morten Magnusson}
\date{November 20, 2025}

\begin{document}
\maketitle

\begin{abstract}
This paper explores the relationship between the speed of light and the two fundamental entropic states in Energy-Flow Cosmology (EFC): the ground state $s_0$, and the active-flow state $s_1$. We outline how light-speed is not a universal constant, but an emergent limit of information transfer set by the entropic gradient between these states. The transition from $s_0$ to $s_1$ defines the available degrees of freedom for energy, information, and curvature. This provides a thermodynamic interpretation of light-speed variability across different cosmic environments without invoking exotic physics.
\end{abstract}

\section{Introduction}

In the EFC framework, the speed of light is not a fixed ontological quantity, but the consequence of underlying thermodynamic structure. The fundamental entropic states, $s_0$ and $s_1$, describe two regimes:

\begin{itemize}
    \item $s_0$: minimal entropy, low available degrees of freedom, near-ground energy configuration.
    \item $s_1$: active-flow regime, increased entropy, structure formation, causal gradients.
\end{itemize}

The transition between these states defines the local information capacity and therefore the upper bound on propagation velocity.

\section{The $s_0$ State: Low-Entropy Ground Configuration}

The $s_0$ state corresponds to a high-coherence, low-entropy regime where curvature is minimal, the energy-flow field is weak, information capacity is low, and degrees of freedom are constrained.

In this state, the effective speed of light approaches a maximum due to minimal entropic drag.

\section{The $s_1$ State: Active Entropic Flow}

The $s_1$ state is characterized by increased entropy, active energy flow, structure formation, curvature, and local increases in information density. Effective light-speed decreases due to entropic resistance.

\section{Variable Light Speed as an Entropic Phenomenon}

Light-speed variability arises from differences in the $s_0 / s_1$ balance:

\[
c_{\text{eff}} = c_0 f(s_0, s_1)
\]

High-$s_1$ regions reduce $c_{\text{eff}}$; high-$s_0$ regions increase it. This produces a natural emergent model without exotic physics.

\section{Implications for Cosmology}

This framework predicts:

\begin{itemize}
    \item Light-speed anisotropy across large-scale structure.
    \item Redshift deviations without expansion-only metrics.
    \item Natural alternatives to dark energy.
    \item Modified lensing time-delays.
    \item Predictable light-speed variation across entropic gradients.
\end{itemize}

\section{Conclusion}

The speed of light emerges from the entropic structure of spacetime. The $s_0/s_1$ states define the local energy-flow environment, shaping the maximum information propagation rate.

\end{document}
