\documentclass[11pt]{article}
\usepackage{amsmath, amssymb}
\usepackage{graphicx}
\usepackage{geometry}
\usepackage{physics}
\usepackage{hyperref}
\usepackage{bm}
\usepackage{tikz}

\geometry{margin=2.5cm}

\title{\textbf{Variable Effective Light Speed in Entropic Transition States:\\
A Formal Treatment of the $s_0$ / $s_1$ Structure in Energy-Flow Cosmology (EFC)}}

\author{Morten Magnusson}
\date{Version 2.0 --- November 2025}

\begin{document}
\maketitle

\begin{abstract}
In Energy-Flow Cosmology (EFC), the effective speed of light is not a primitive
constant but an emergent propagation limit determined by the entropic state of
the grid. Two fundamental entropic phases---a low-entropy constraining state
($s_0$) and a high-entropy dissipative state ($s_1$)---govern the information
capacity, local curvature, and energy-flow geometry. This paper provides a
formal description of variable light speed as a function of the entropic
gradient, derives a minimal model for $c_{\mathrm{eff}}(S)$, and introduces the
collapse mechanisms at both ends of the spectrum that shape observational
horizons and redshift behaviour.
\end{abstract}

\section{Introduction}
Energy-Flow Cosmology (EFC) describes spacetime as a coupled
energy--entropy--information grid. Geometry is not fundamental: curvature and
signal propagation emerge from the dynamic flow of energy and the distribution
of entropy across the grid.

Instead of treating the speed of light as an axiom, we view the \emph{effective}
speed of light $c_{\mathrm{eff}}$ as the maximum information-propagation rate in
a local entropic context. The structure is dominated by two coarse-grained
states:
\begin{itemize}
\item $s_0$: a low-entropy, low-information, near-rigid state;
\item $s_1$: a high-entropy, high-information, turbulent state.
\end{itemize}

\section{The $s_0$ State: Low-Entropy Rigidity}
The $s_0$ state is defined by:
\begin{itemize}
\item minimal entropy production,
\item weak or nearly uniform energy-flow fields,
\item few active degrees of freedom,
\item low curvature and near-rigid grid behaviour.
\end{itemize}

Near the $s_0$ limit, the grid becomes too constrained. Propagation encounters
\emph{structural rigidity}, where local adjustments are suppressed. As the system
approaches the $s_0$-collapse threshold, accumulated tension is released in a
coherence ``pop'' event. Just before collapse, $c_{\mathrm{eff}}$ increases sharply
due to a transient drop in entropic drag.

\section{The $s_1$ State: High-Entropy Dissipation}
The $s_1$ state exhibits:
\begin{itemize}
\item high entropy and strong entropy production,
\item high information density,
\item turbulent and structured energy flows,
\item enhanced curvature and deformation.
\end{itemize}

Near the $s_1$ extreme, propagation loses coherence. Excessive fluctuations
obstruct signal propagation; entropic drag dominates and $c_{\mathrm{eff}}$
falls toward zero. Collapse occurs when coherent information channels fail.

\section{A Bell-Shaped Model for $c_{\mathrm{eff}}(S)$}
Let $S$ represent the local entropic state. The effective speed of light reaches
a maximum at an intermediate optimal value $S_{\ast}$ where the grid balances
flexibility and stability.

A minimal symmetric model is:
\begin{equation}
    c_{\mathrm{eff}}(S) = 
    c_0 \exp\!\left[-\alpha (S - S_{\ast})^2 \right],
\end{equation}
where $c_0$ is the maximum propagation speed and $\alpha$ sets the width.

\begin{figure}[h!]
\centering
\begin{tikzpicture}[scale=1.2]
    % Axes
    \draw[->] (-0.2,0) -- (6.2,0) node[right] {$S$};
    \draw[->] (0,-0.2) -- (0,3.2) node[above] {$c_{\mathrm{eff}}$};

    % Bell curve
    \draw[thick,blue,domain=0.3:5.7,samples=200]
        plot (\x,{3*exp(-0.5*(\x-3)^2)});

    % Labels
    \draw (0.8,-0.15) node {$s_0$};
    \draw (3,-0.15) node {$S_{\ast}$};
    \draw (5.2,-0.15) node {$s_1$};

    % Vertical lines
    \draw[dashed] (0.8,0) -- (0.8,1.1);
    \draw[dashed] (3,0) -- (3,3);
    \draw[dashed] (5.2,0) -- (5.2,1.1);

    % Peak label
    \draw (3,3.2) node {$\text{maximum } c_{\mathrm{eff}}$};
\end{tikzpicture}
\caption{Bell-shaped model for the effective propagation speed $c_{\mathrm{eff}}$ as a function
of the entropic state $S$. The grid transitions from rigid ($s_0$) to optimal
flexibility ($S_{\ast}$) and then to dissipative ($s_1$).}
\end{figure}

\section{Collapse Dynamics}
\subsection{$s_0$-Collapse}
Low entropy produces rigidity. Stress accumulates until the grid can no longer
maintain coherence; collapse releases this in a high-coherence spike with a
temporary rise in $c_{\mathrm{eff}}$.

\subsection{$s_1$-Collapse}
High entropy destroys coherence. Excessive microscopic variation blocks
propagation. As $c_{\mathrm{eff}} \rightarrow 0$, the grid loses the capacity to
transmit information. This forms a natural observational boundary.

\section{Redshift as an Entropic Integral}
For a photon:
\[
E_\gamma = h\nu, 
\qquad 
\nu = \frac{c_{\mathrm{eff}}}{\lambda}.
\]

As $c_{\mathrm{eff}}$ decreases in $s_1$-dominated regions:
\begin{itemize}
\item $\nu$ decreases,
\item $E_\gamma$ decreases,
\item the photon is redshifted.
\end{itemize}

The general form is:
\begin{equation}
z = \int_\gamma 
g\!\left(S(x), \nabla S(x), c_{\mathrm{eff}}(S)\right) ds,
\end{equation}
where $\gamma$ is the photon path and $g$ encodes local entropic drag.

This explains:
\begin{itemize}
\item extreme high-$z$ galaxies (low $c_{\mathrm{eff}}$ regions),
\item the CMB as an $s_1$-collapse boundary,
\item observational limits without invoking metric expansion.
\end{itemize}

\section{Cosmological Implications}
\begin{itemize}
\item coherent explanation of high redshift without accelerated expansion,
\item anisotropies tied to entropy structure,
\item modified lensing time delays,
\item early galaxy formation via $S$-gradient effects,
\item natural horizon from $s_1$-collapse.
\end{itemize}

\section{Conclusion}
In EFC, the speed of light is an emergent, entropic quantity. The $s_0/s_1$
framework provides a minimal description of how information propagates and why
observational boundaries arise. This model replaces a rigid constant $c$ with a 
state-dependent limit tied directly to energy flow and entropy.

\end{document}
