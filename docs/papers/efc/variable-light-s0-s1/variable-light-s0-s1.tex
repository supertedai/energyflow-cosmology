\documentclass[12pt]{article}

\usepackage[a4paper, margin=2.5cm]{geometry}
\usepackage{amsmath, amssymb}
\usepackage[T1]{fontenc}
\usepackage[utf8]{inputenc}
\usepackage{hyperref}

\title{Variable Light Speed and the s\textsubscript{0} / s\textsubscript{1} Entropic States}
\author{Morten Magnusson}
\date{\today}

\begin{document}
\maketitle

\begin{abstract}
This note situates variable light speed inside the Energy-Flow Cosmology (EFC) framework. Instead of treating $c$ as a primitive constant, we view the effective speed of light as an emergent property of the entropic transition between two fundamental states: a low-entropy ground configuration $s_0$ and an active flow state $s_1$. The $s_0 / s_1$ structure constrains information capacity, energy-flow geometry, and thus the maximum propagation rate of signals.
\end{abstract}

\section{Motivation}

Within EFC, the universe is described as a thermodynamic flow system. Energy, entropy, and information capacity are primary; geometry and observed constants emerge from them. In this view, the speed of light is not an isolated postulate but the observable limit of information transfer in a given entropic environment.

Two coarse-grained states are central:

\begin{itemize}
  \item $s_0$: a near-ground, low-entropy, low-curvature configuration with few active degrees of freedom.
  \item $s_1$: an excited, high-entropy, structured configuration with active energy flow and curvature.
\end{itemize}

The transition between these states controls the local causal structure.

\section{The $s_0$ state}

The $s_0$ state is characterized by:

\begin{itemize}
  \item minimal entropy production,
  \item weak or nearly uniform energy-flow fields,
  \item low information density,
  \item weak curvature in the effective grid.
\end{itemize}

In this regime, entropic drag on propagation is minimal, and the effective light speed $c_{\mathrm{eff}}$ approaches a maximal value $c_0$ set by the underlying grid-scale dynamics.

\section{The $s_1$ state}

The $s_1$ state is characterized by:

\begin{itemize}
  \item higher entropy and entropy production,
  \item strong, structured energy flows,
  \item increased information density,
  \item locally enhanced curvature and distortion in the grid.
\end{itemize}

Here, propagation interacts with gradients and structure. The effective light speed is reduced compared to $s_0$ because signals traverse an entropically ``rough'' medium.

\section{An entropic view of $c_{\mathrm{eff}}$}

At a coarse level we can write
\[
  c_{\mathrm{eff}} = c_0 \, f(s_0, s_1),
\]
where $f$ encodes how the local balance between $s_0$ and $s_1$ constrains propagation. Regions dominated by $s_0$ (high coherence, low entropy) yield $f \approx 1$, while strongly $s_1$-dominated regions yield $f < 1$.

In the full EFC picture, $f$ is not a free function but is tied to:

\begin{itemize}
  \item the local energy-rate density,
  \item the available information capacity,
  \item the entropic gradient across the grid,
  \item the halo and cluster-scale flow structures.
\end{itemize}

\section{Cosmological implications}

An entropic, state-dependent $c_{\mathrm{eff}}$ implies:

\begin{itemize}
  \item light-speed anisotropy across different large-scale environments,
  \item redshift behaviour that couples to entropy and flow, not bare expansion alone,
  \item modified time delays in lensing and cluster environments,
  \item a route to explain some observations attributed to dark energy or exotic modifications.
\end{itemize}

These are not introduced as separate mechanisms but fall out of how $s_0 / s_1$ structure the grid.

\section{Conclusion}

In EFC, the speed of light is not a sacred constant but an emergent limit shaped by entropic state. The $s_0$ and $s_1$ states form the minimal language for describing how information moves through the grid. Variable light speed then becomes a diagnostic of underlying entropic structure, rather than a violation of relativity.

\end{document}
