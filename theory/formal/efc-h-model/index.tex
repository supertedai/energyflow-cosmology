% =====================================================================
% Energy-Flow Cosmology — EFC-H Structural Model (Master Specification)
% =====================================================================

\section*{EFC–H Structural Model: Formal Specification}

\subsection*{1. Purpose}

The EFC–H model defines how stable structures emerge from the interaction
between entropy, structural density, energy flow, and grid resistance in
Energy-Flow Cosmology (EFC). It describes halo formation, structure
profiles, and long-range stability produced by the energy–entropy system.

\subsection*{2. Core Fields}

\begin{itemize}
    \item Entropy field: $S(\mathbf{x}, t)$
    \item Structural density: $\rho(\mathbf{x}, t)$
    \item Energy–flow field: $E_f(\mathbf{x}, t)$
    \item Grid resistance: $R(\mathbf{x}, t)$
    \item Effective propagation speed: $c_{\text{eff}}$
\end{itemize}

\subsection*{3. Structural Formation Condition}

A region becomes a stable structure when:

\[
E_f > R.
\]

Equivalently:

\[
\rho(1 - S) > R(S, \rho).
\]

This defines the **structural boundary condition**.

\subsection*{4. Structural Growth Law}

Density evolves according to:

\[
\frac{d\rho}{dt} = F(E_f, \nabla E_f),
\]

with a minimum closure:

\[
\frac{d\rho}{dt} \propto E_f - R.
\]

If $E_f > R$, structure condenses;  
if $E_f < R$, structure disperses.

\subsection*{5. Halo Profile Equation}

For quasi-static structures:

\[
\nabla E_f = 0,
\]

which yields:

\[
\rho(1-S) = \text{constant}.
\]

This defines the core **EFC halo law**, linking:

- low local entropy  
- moderate to high structure  
- stable energy-flow equilibrium  

\subsection*{6. Gradient-Coupled Structure}

Spatial gradients define limb, transition, and asymptotic regimes:

\[
\nabla S(\mathbf{x}) \neq 0
\quad \Rightarrow \quad
\text{extended halo behaviour}.
\]

The extended halo region satisfies:

\[
E_f(\mathbf{x}) \approx \text{constant},
\]

while $S(\mathbf{x})$ slowly increases with radius.

\subsection*{7. Resistance–Structure Coupling}

Resistance reduces as structure grows:

\[
\frac{\partial R}{\partial \rho} < 0,
\]

and increases with entropy:

\[
\frac{\partial R}{\partial S} > 0.
\]

This coupling produces:

- stable cores (low $S$, high $\rho$, low $R$)  
- extended halos (moderate $S$, moderate $R$)  
- fading outskirts (higher $S$, higher $R$)

\subsection*{8. Effective Propagation Velocity}

Propagation speed is:

\[
c_{\text{eff}} = (1 + R)^{-1}.
\]

Inside structures ($R$ low), propagation is faster;  
outside ($R$ high), propagation is slower.

This produces observable lensing and apparent mass enrichment
without dark matter.

\subsection*{9. Full EFC–H System}

\[
\begin{aligned}
E_f &= \rho(1 - S), \\
\frac{d\rho}{dt} &= F(E_f, \nabla E_f), \\
R &= R(S, \rho), \\
c_{\text{eff}} &= (1 + R)^{-1}, \\
\nabla E_f &\rightarrow \text{halo profiles}, \\
E_f > R &\rightarrow \text{structural condensation}.
\end{aligned}
\]

\subsection*{10. Role in the EFC Framework}

EFC–H provides:

\begin{itemize}
    \item the base for halo formation without dark matter
    \item a structural layer linking dynamics (EFC–D) to density profiles
    \item the formal underpinning for cluster-scale predictions and CMB distortions
    \item the foundation for the EFC–S structural module
\end{itemize}

EFC–H is the core structural layer of EFC, providing the theoretical
basis for all emergent mass, halo shapes, and gravitational-like effects.
