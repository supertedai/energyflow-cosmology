% ============================================================
% EFC Formal Specification
% ============================================================

\section*{EFC Formal Specification}

This document defines the core mathematical structure of
Energy–Flow Cosmology (EFC). All higher-level models, diagrams
and simulation modules reference the definitions given here.

% ------------------------------------------------------------
\subsection*{1. Fields}

\paragraph{Entropy field}
\[
S = S(\mathbf{x})
\]

\paragraph{Density field}
\[
\rho = \rho(\mathbf{x})
\]

\paragraph{Grid resistance}
\[
R = R(\mathbf{x})
\]

\paragraph{Energy–flow potential}
\[
E_f = \rho \, (1 - S)
\]

% ------------------------------------------------------------
\subsection*{2. Derived Quantities}

\paragraph{Local gradient of entropy}
\[
\nabla S
\]

\paragraph{Temporal change of energy–flow}
\[
\frac{dE_f}{dt} = - \nabla S \cdot \nabla E_f
\]

\paragraph{Grid-induced propagation resistance}
\[
c_\mathrm{eff} = c_0 \, (1 - S)
\]

% ------------------------------------------------------------
\subsection*{3. Structural Relationships}

\begin{itemize}
  \item Low entropy increases \(E_f\)
  \item High density increases \(E_f\)
  \item Grid resistance modulates propagation speed
  \item Feedback loops: \(S \leftrightarrow E_f \leftrightarrow \rho\)
\end{itemize}

% ------------------------------------------------------------
\subsection*{4. Role in EFC}

This formal specification provides:
\begin{itemize}
  \item A single coherent mathematical baseline
  \item Compatibility across EFC-S, EFC-D and EFC-C
  \item A stable foundation for numerical simulations
\end{itemize}

\end{document}
