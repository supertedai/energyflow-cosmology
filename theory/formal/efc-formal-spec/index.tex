% =====================================================================
% Energy-Flow Cosmology — Formal Specification
% =====================================================================

\section*{Energy-Flow Cosmology: Formal Specification}

\subsection*{1. Fields and Core Quantities}

Let the spatial domain be described by coordinates $\mathbf{x}$.

\begin{itemize}
    \item Entropy field: $S(\mathbf{x})$
    \item Structural density: $\rho(\mathbf{x})$
    \item Energy–flow potential: $E_f(\mathbf{x})$
    \item Grid resistance (propagation resistance): $R(\mathbf{x})$
    \item Information capacity: $C(\mathbf{x})$
\end{itemize}

Entropy is normalized:
\[
0 \leq S(\mathbf{x}) \leq 1.
\]

The local density is:
\[
\rho(\mathbf{x}) = \frac{m}{V}.
\]

\subsection*{2. Energy–Flow Potential}

The core energy–flow potential is defined as:
\[
E_f = \rho\, (1 - S).
\]

High density and low entropy increase $E_f$; high entropy suppresses it.

The temporal change of energy–flow is:
\[
\frac{dE_f}{dt} = \nabla_t E_f.
\]

\subsection*{3. Entropy Gradient and Flow}

The entropy gradient drives changes in energy distribution:

\[
\nabla S(\mathbf{x}) \rightarrow \text{redistribution of } E_f.
\]

Regions with steep entropy gradients produce strong modulations in $E_f$.

\subsection*{4. Structural Response}

Structure arises where energy–flow stabilizes:

\[
\frac{d\rho}{dt} = F(E_f, \nabla E_f).
\]

A simplified local approximation:

\[
\frac{d\rho}{dt} \propto E_f - R(\mathbf{x}),
\]

where $R(\mathbf{x})$ encodes resistance from the grid state.

\subsection*{5. Propagation Resistance}

Propagation resistance affects how energy and information move:

\[
R = R(S, \rho).
\]

Higher entropy increases resistance; higher structure decreases it.

The effective propagation speed (e.g., of light or information) follows:

\[
c_{\text{eff}} = \frac{1}{1 + R}.
\]

\subsection*{6. Information Capacity}

Information capacity maps entropy to local representational bandwidth:

\[
C(\mathbf{x}) = C_0 \, (1 - S(\mathbf{x})).
\]

Low entropy increases representational capacity; high entropy reduces it.

\subsection*{7. Coupled System}

The EFC base system is:

\[
\begin{aligned}
E_f &= \rho\,(1-S), \\
\frac{dE_f}{dt} &= \nabla_t E_f, \\
\frac{d\rho}{dt} &= F(E_f, \nabla E_f), \\
R &= R(S,\rho), \\
c_{\text{eff}} &= (1 + R)^{-1}, \\
C &= C_0\,(1-S).
\end{aligned}
\]

This forms the minimal closed specification connecting entropy, energy–flow,
structure, resistance and information capacity.

\subsection*{8. Purpose of the Specification}

This formal specification is the reference for all higher-level EFC models,
including:

\begin{itemize}
    \item EFC-S (structural halo profiles)
    \item EFC-D (global and local dynamics)
    \item EFC-C$_0$ (entropy–information mapping)
    \item Simulation modules (CLASS, RAMSES, GADGET-4 adaptations)
\end{itemize}

It provides a mathematically explicit baseline that ensures internal consistency
across the theory.
