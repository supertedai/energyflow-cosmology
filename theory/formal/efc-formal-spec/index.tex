\documentclass[a4paper,11pt]{article}

% -------------------------------------------------------
% Packages
% -------------------------------------------------------
\usepackage{amsmath,amssymb}
\usepackage{bm}
\usepackage{geometry}
\usepackage{hyperref}
\usepackage{physics}
\usepackage{graphicx}
\geometry{margin=2.5cm}

% -------------------------------------------------------
% Document
% -------------------------------------------------------
\begin{document}

\title{\textbf{Energy-Flow Cosmology (EFC)\\Formal Specification}}
\author{Morten Magnusson}
\date{\today}
\maketitle

\tableofcontents
\newpage

% -------------------------------------------------------
\section{Introduction}

This document defines the formal mathematical structure of
Energy-Flow Cosmology (EFC). The formal specification provides the
core definitions, relationships, and functional laws that govern the
entropy field $S(\mathbf{x},t)$, the energy–flow field $E_f(\mathbf{x},t)$,
and the grid-level resistance functions that shape the behaviour of
spacetime and all emergent structures.

The formal specification consists of five components:
\begin{itemize}
    \item The EFC--S structural model
    \item The EFC--D dynamical model
    \item The EFC--C0 information model
    \item The EFC--H grid--Higgs interaction model
    \item Shared notation and parameter definitions
\end{itemize}

Each component has its own source definitions within the repository, but
this master specification functions as the structural anchor.

% -------------------------------------------------------
\section{Field Definitions}

\subsection{Entropy Field $S$}

The entropy field $S(\mathbf{x},t)$ is defined as a scalar field taking
values in the interval $[0,1]$.

\begin{equation}
    S : \mathbb{R}^3 \times \mathbb{R} \rightarrow [0,1]
\end{equation}

Two limiting boundary states are defined:
\[
S = 0 \quad \text{(minimum entropy state)}
\]
\[
S = 1 \quad \text{(maximum entropy state)}
\]

The gradient of $S$ defines the entropic directionality:

\begin{equation}
    \nabla S(\mathbf{x},t) = 
    \left(
    \frac{\partial S}{\partial x},
    \frac{\partial S}{\partial y},
    \frac{\partial S}{\partial z}
    \right)
\end{equation}

\subsection{Energy-Flow Field $E_f$}

Energy flow is represented by the vector field:

\begin{equation}
    E_f : \mathbb{R}^3 \times \mathbb{R} \rightarrow \mathbb{R}^3
\end{equation}

The fundamental relationship between the fields is:

\begin{equation}
    E_f = -\kappa \, \nabla S
\end{equation}

where $\kappa$ is the grid-level resistance coefficient.

% -------------------------------------------------------
\section{EFC--S Structural Model}

The spatial distribution of $S$ determines the emergent structural
patterns. The EFC--S model defines the scalar curvature analogue
$C_S$:

\begin{equation}
    C_S = \nabla \cdot (\nabla S)
\end{equation}

Regions where $C_S < 0$ correspond to convergent entropic basins;
regions with $C_S > 0$ correspond to divergent entropic regions.

% -------------------------------------------------------
\section{EFC--D Dynamical Model}

The dynamical behaviour of $S$ evolves according to:

\begin{equation}
    \frac{\partial S}{\partial t}
    = -\alpha \, \nabla \cdot E_f
    + \beta \, \Delta S
\end{equation}

where:
\begin{itemize}
    \item $\alpha$ controls flow–driven change
    \item $\beta$ controls diffusive smoothing
\end{itemize}

This is the primary governing equation for large-scale EFC dynamics.

% -------------------------------------------------------
\section{EFC--C0 Information Model}

The C0 model defines the informational resonance functional:

\begin{equation}
    I_0 = \gamma \, \left| E_f \cdot \nabla S \right|
\end{equation}

$ I_0 $ is interpreted as the local information flux density.

% -------------------------------------------------------
\section{EFC--H Grid-Higgs Interaction}

Mass–like behaviour emerges from resistance to energy flow:

\begin{equation}
    m_{\mathrm{eff}} = \lambda \, \| E_f \|
\end{equation}

where $\lambda$ is the effective coupling parameter.

This replaces particle–Higgs mass generation with a flow-resistance model.

% -------------------------------------------------------
\section{Parameter Summary}

\begin{itemize}
    \item $\kappa$ – grid resistance coefficient
    \item $\alpha$ – flow amplification coefficient
    \item $\beta$ – entropic diffusion coefficient
    \item $\gamma$ – information coupling coefficient
    \item $\lambda$ – grid–Higgs effective coupling
\end{itemize}

These parameters are defined in more detail in
\texttt{theory/formal/parameters/index.tex}.

% -------------------------------------------------------
\section{Notation}

Notation conventions are centralised in
\texttt{theory/formal/notation/index.tex}.

% -------------------------------------------------------
\section{Conclusion}

This master formal specification defines the core mathematical
foundation of the Energy-Flow Cosmology framework. All model-specific
definitions extend from the structures defined in this document.

\end{document}
