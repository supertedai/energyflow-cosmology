% =========================================================================
% Energy-Flow Cosmology — Formal Header Specification
% =========================================================================

\section*{EFC Formal Header}

\subsection*{1. Purpose}

The EFC Formal Header defines the shared notation, global constants,
normalizations, and macro definitions used across all formal modules
(EFC–S, EFC–D, EFC–H, EFC–C$_0$, parameters, diagrams, and specifications).

It provides a consistent symbolic base and ensures uniform mathematical
presentation across the entire EFC theory.

\subsection*{2. Global Normalization}

Entropy is defined over a normalized domain:

\[
0 \leq S(\mathbf{x}, t) \leq 1.
\]

Energy–flow potential:

\[
E_f = \rho(1-S).
\]

Propagation speed normalization:

\[
c_{\text{eff}} = (1 + R)^{-1}.
\]

\subsection*{3. Shared Symbols}

\begin{itemize}
    \item $S$ — entropy field  
    \item $\rho$ — structural density  
    \item $E_f$ — energy–flow field  
    \item $R$ — grid resistance  
    \item $C$ — information capacity  
    \item $I_0$ — information resonance scalar  
    \item $c_{\text{eff}}$ — effective propagation speed  
\end{itemize}

\subsection*{4. Standard LaTeX Macros}

\begin{verbatim}
\newcommand{\Ef}{E_f}
\newcommand{\dS}{\frac{dS}{dt}}
\newcommand{\drho}{\frac{d\rho}{dt}}
\newcommand{\dEf}{\frac{dE_f}{dt}}
\newcommand{\dR}{\frac{dR}{dt}}
\newcommand{\ceff}{c_{\text{eff}}}
\end{verbatim}

\subsection*{5. Spatial Notation}

\[
\nabla S, \quad \nabla E_f, \quad \nabla \rho
\]

represent local spatial gradients.

Temporal derivatives:

\[
\frac{d}{dt} = \frac{\partial}{\partial t} + \mathbf{v}\cdot\nabla.
\]

\subsection*{6. Diagrams and Flow Ordering}

The canonical EFC dependency chain:

\[
S \rightarrow \rho \rightarrow E_f \rightarrow R \rightarrow c_{\text{eff}}
\]

is defined here as a global convention.

\subsection*{7. Purpose in the EFC Framework}

The header:

\begin{itemize}
    \item harmonizes notation across all formal modules
    \item ensures model compatibility
    \item simplifies LaTeX compilation
    \item supports machine-readable mapping of symbols and meaning
\end{itemize}

It is included or referenced by all higher-level specifications.
