\subsection{Effective light speed as a derived quantity}

In EFC, the speed of light is not a universal constant. It is an
\emph{effective} propagation speed that emerges from the entropy field
$S$, the s$_0$--s$_1$ endpoint structure, and the resistance of the
energy–flow grid.

\paragraph{Definition.}

Let $S_0$ and $S_1$ denote the endpoint entropies. Define
\[
S_{\text{mid}} = \frac{1}{2}(S_0 + S_1),
\qquad
\Delta S = S_1 - S_0,
\qquad
x(S) = \frac{S - S_{\text{mid}}}{\Delta S / 2}.
\]

The effective speed of light is then
\begin{equation}
    c(S) = c_0 \left( 1 + a_{\text{edge}} \, x(S)^2 \right),
\end{equation}
with $c_0$ the baseline mid-entropy speed and $a_{\text{edge}} > 0$ a
dimensionless parameter controlling enhancement toward the endpoints.

\paragraph{Interpretation.}

\begin{itemize}
    \item $c(S)$ is minimal in the mid-entropy regime.
    \item $c(S)$ increases smoothly as $S \to S_0$ (focusing).
    \item $c(S)$ increases smoothly as $S \to S_1$ (defocusing).
\end{itemize}

\paragraph{Implications.}

Light propagation depends on the path integral
\[
t_{\text{obs}} = \int_{\gamma} \frac{dl}{c(S(l))},
\]
meaning that all observable quantities that rely on null propagation
(time delays, redshift–distance relations, lensing patterns) must be
computed using $c(S)$ rather than a fixed constant $c$.

This entropy-dependent propagation law is therefore a central element
in the formal EFC framework.
