\documentclass[a4paper,11pt]{article}

% -------------------------------------------------------
% Packages
% -------------------------------------------------------
\usepackage{amsmath,amssymb}
\usepackage{graphicx}
\usepackage{hyperref}
\usepackage{physics}
\usepackage{bm}
\usepackage{geometry}
\geometry{margin=2.5cm}

% -------------------------------------------------------
% Document
% -------------------------------------------------------
\begin{document}

\title{Energy-Flow Cosmology (EFC)\\
Formal Specification}
\author{Morten Magnusson}
\date{\today}
\maketitle

\tableofcontents
\newpage

% -------------------------------------------------------
\section{Introduction}

Energy-Flow Cosmology (EFC) is a thermodynamic framework in which
structure, dynamics, and observable propagation all emerge from the
interaction between the entropy field $S$, the energy--flow field $E_f$,
and the grid-level resistance that shapes how energy and light move
through spacetime.

This document defines the mathematical objects, structural relationships,
and dynamical laws that form the formal specification of EFC.

% -------------------------------------------------------
\section{Foundations of EFC}

The two core scalar fields of EFC are:

\begin{itemize}
    \item the entropy field $S(\mathbf{x})$;
    \item the energy--flow field $E_f(\mathbf{x})$.
\end{itemize}

All dynamical and observable quantities are derived from these.

% -------------------------------------------------------
\subsection{Entropy Field}

The entropy field $S(\mathbf{x})$ encodes the resistance of the grid and
defines the focusing and defocusing tendencies of both energy and light.

Two dynamic endpoints bound the entropy domain:

\[
S_0 \quad \text{(s$_0$: low entropy, high structure)}, \qquad
S_1 \quad \text{(s$_1$: high entropy, structural loosening)}.
\]

The midpoint and entropy span are:

\[
S_{\text{mid}} = \frac{1}{2}(S_0 + S_1),
\qquad
\Delta S = S_1 - S_0.
\]

% -------------------------------------------------------
\section{Dynamical Sector (EFC-D)}

The energy-flow field is defined as:

\[
E_f : \mathbb{R}^3 \rightarrow \mathbb{R}.
\]

It describes the directional flow of energy along entropy gradients.

\subsection{Energy-Flow Potential}

The effective potential that governs structural and dynamical behavior is:

\[
\Phi(E_f,S) = A_\Phi \, E_f \, (1 + S).
\]

More general couplings such as

\[
\Phi = A_\Phi E_f (1 + S^\beta)
\]

may be introduced but are not implemented in the baseline model.

\subsection{Circular Velocity Relation}

Rotation curves follow:

\[
v(r) = \sqrt{\, r \, \frac{\partial \Phi}{\partial r} \, }.
\]

Here the observed behavior depends both on $\Phi$ and the entropy-modified
light propagation defined in the next section.

% -------------------------------------------------------
\section{Light Propagation in EFC}

\subsection{Effective light speed as a derived quantity}

In EFC, the speed of light is not a universal constant. It is an
\emph{effective} propagation speed that emerges from the entropy field
$S$, the s$_0$--s$_1$ endpoint structure, and the resistance of the
energy–flow grid.

\paragraph{Definition.}

Let $S_0$ and $S_1$ denote the endpoint entropies. Define
\[
S_{\text{mid}} = \frac{1}{2}(S_0 + S_1),
\qquad
\Delta S = S_1 - S_0,
\qquad
x(S) = \frac{S - S_{\text{mid}}}{\Delta S / 2}.
\]

The effective speed of light is then
\begin{equation}
    c(S) = c_0 \left( 1 + a_{\text{edge}} \, x(S)^2 \right),
\end{equation}
with $c_0$ the baseline mid-entropy speed and $a_{\text{edge}} > 0$ a
dimensionless parameter controlling enhancement toward the endpoints.

\paragraph{Interpretation.}

\begin{itemize}
    \item $c(S)$ is minimal in the mid-entropy regime.
    \item $c(S)$ increases smoothly as $S \to S_0$ (focusing).
    \item $c(S)$ increases smoothly as $S \to S_1$ (defocusing).
\end{itemize}

\paragraph{Implications.}

Light propagation depends on the path integral
\[
t_{\text{obs}} = \int_{\gamma} \frac{dl}{c(S(l))},
\]
meaning that all observable quantities that rely on null propagation
(time delays, redshift–distance relations, lensing patterns) must be
computed using $c(S)$ rather than a fixed constant $c$.

This entropy-dependent propagation law is therefore a central element
in the formal EFC framework.

% -------------------------------------------------------
\subsection{Light Travel Times and Lensing}

The observable travel time of a photon following trajectory $\gamma$ is:

\[
t_{\text{obs}} = \int_{\gamma} \frac{dl}{c(S(l))}.
\]

Consequences include:

\begin{itemize}
    \item faster propagation near s$_0$ and s$_1$ (enhanced $c(S)$);
    \item modified magnification and distortion patterns;
    \item altered time-delay ratios even with identical mass distributions;
    \item redshift--distance relations dependent on the entropy field.
\end{itemize}

% -------------------------------------------------------
\section{Structural Sector (EFC-S)}

Structures in EFC arise from the interplay of $S$, $E_f$, $\Phi(E_f,S)$,
and the effective light speed $c(S)$. The mid-entropy region where
$c(S)$ is minimal forms a natural stability band for halo-like
configurations.

% -------------------------------------------------------
\section{Conclusion}

This formal specification defines the mathematical backbone of
Energy-Flow Cosmology: the entropy field, the energy-flow field, the
effective potential, and the emergent propagation law $c(S)$ that shapes
all null-propagation observables.

\end{document}
