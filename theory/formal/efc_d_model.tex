\section{EFC-D: Energy-Flow Dynamics}

The dynamical sector of Energy-Flow Cosmology (EFC-D) describes how
energy propagates through the grid defined by the entropy field $S$ and
the energy–flow field $E_f(\mathbf{x})$. All dynamical behaviour
(gravitational focusing, late-time acceleration, structural stability,
and light propagation) emerges from the coupling between $S$, $E_f$,
and the effective speed of light $c(S)$.

% ---------------------------------------------------------
\subsection{Energy-Flow Field}

We define the energy–flow field as
\begin{equation}
    E_f : \mathbb{R}^3 \rightarrow \mathbb{R}.
\end{equation}

Physically, $E_f$ measures how strongly energy is channeled along entropy
gradients. Regions with strong gradients in $S$ create guided-flow
structures that influence both matter motion and photon propagation.

% ---------------------------------------------------------
\subsection{Effective Speed of Light $c(S)$}

In EFC, the speed of light is not a universal constant but an
\emph{effective} propagation speed determined by the grid resistance,
which depends on the entropy field $S$ and the s$_0$--s$_1$ endpoint
structure.

Let $S_0$ and $S_1$ denote the entropy values associated with the
dynamic endpoints s$_0$ (low entropy, high structure) and s$_1$
(high entropy, structure collapse). Define:
\[
S_{\text{mid}} = \frac{1}{2}(S_0 + S_1), 
\qquad
\Delta S = S_1 - S_0.
\]

A normalized entropy coordinate:
\[
x(S) = \frac{S - S_{\text{mid}}}{\Delta S / 2},
\]
gives $x=-1$ at s$_0$, $x=0$ at the midpoint, and $x=+1$ at s$_1$.

The effective speed of light is then:
\begin{equation}
    c(S) = c_0 \left( 1 + a_{\text{edge}} \, x(S)^2 \right),
\end{equation}
where $c_0$ is the baseline mid-range light speed and 
$a_{\text{edge}} > 0$ regulates the endpoint enhancement.

Key properties:
\begin{itemize}
    \item $c(S)$ is \emph{minimum} at the entropy midpoint.
    \item $c(S)$ increases smoothly as $S \to S_0$ (s$_0$-side focusing).
    \item $c(S)$ also increases as $S \to S_1$ (s$_1$-side defocusing).
\end{itemize}

Thus, light propagation, time delays, and apparent dynamical behaviour
must be treated as functionals of $c(S)$, not a fixed constant $c$.

% ---------------------------------------------------------
\subsection{Energy-Flow Potential}

The effective potential governing matter and energy transport is
\begin{equation}
    \Phi(E_f,S) = A_\Phi \, E_f \, (1 + S),
\end{equation}
which is the baseline form used in the Python implementation,
\[
\Phi \propto E_f S + E_f.
\]

A more general family (not yet implemented) is
\begin{equation}
    \Phi = A_\Phi E_f (1 + S^\beta),
\end{equation}
with $\beta$ controlling non-linear coupling strength.

The coupling to $c(S)$ becomes relevant when computing dynamical
quantities that depend on characteristic propagation speeds
(e.g.\ effective horizon scales, entropy-damped flow rates,
and null trajectories).

% ---------------------------------------------------------
\subsection{Circular Velocity Relation}

For rotation curves, the circular velocity is:
\begin{equation}
    v(r) =
    \sqrt{
        r \frac{\partial \Phi}{\partial r}
    }.
\end{equation}

In EFC-D, this expression is formally unchanged, but its interpretation
is modified by the presence of the entropy-dependent propagation speed
$c(S)$, which influences observed velocities through focusing/defocusing
of photon trajectories and the inferred gravitational field.

Regions where $S(r)$ shifts toward s$_0$ or s$_1$ effectively modify
light propagation and the appearance of rotation curves without invoking
dark matter or additional free parameters.
