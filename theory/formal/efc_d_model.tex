\section{EFC-D: Energy-Flow Dynamics}

The energy-flow field $E_f(\mathbf{x})$ represents the directional
flow of energy across entropy gradients.

\subsection{Energy-Flow Field}

We define:
\begin{equation}
    E_f : \mathbb{R}^3 \rightarrow \mathbb{R}.
\end{equation}

Physically, $E_f$ measures how strongly energy is "channeled" along
structures created by gradients in $S$.

\subsection{Energy-Flow Potential}

EFC defines an effective potential:
\begin{equation}
    \Phi(E_f,S) = A_\Phi \, E_f \, (1 + S).
\end{equation}

This is the baseline form used in the Python implementation:
\begin{equation}
    \Phi \propto E_f S + E_f.
\end{equation}

More general forms (not yet implemented) include:
\begin{equation}
    \Phi = A_\Phi E_f (1 + S^\beta),
\end{equation}
where $\beta$ controls non-linear coupling.

\subsection{Circular Velocity Relation}

For rotation curves:
\begin{equation}
    v(r) = 
    \sqrt{
        r \frac{\partial \Phi}{\partial r}
    }.
\end{equation}

This connects the potential directly to SPARC data.
