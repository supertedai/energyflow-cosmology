\section{EFC-D: Dynamical Sector}

The dynamical sector EFC-D describes how the energy-flow field $E_f$,
the effective potential $\Phi(E_f,S)$ and the expansion rate $H$ emerge
from the entropy field $S$ and its gradients. It is the bridge between
thermodynamics and the observable evolution of structures.

% -------------------------------------------------------
\subsection{Energy-Flow Field}

We define the scalar energy-flow field
\[
E_f : \mathbb{R}^3 \rightarrow \mathbb{R},
\]
which encodes the strength and directionality of energy transport along
entropy gradients. In regions with strong gradients $\nabla S$, the
magnitude $|E_f|$ is large, while in nearly uniform regions $E_f$
approaches zero.

We assume that $E_f$ is aligned with $-\nabla S$ at leading order,
\[
E_f \propto - \nabla S,
\]
reflecting that energy tends to flow from low- to high-entropy regions.

% -------------------------------------------------------
\subsection{Effective Potential \texorpdfstring{$\Phi(E_f,S)$}{Phi(Ef,S)}}

The effective potential is defined as
\[
\Phi(E_f,S) = A_\Phi \, E_f \, (1 + S),
\]
where $A_\Phi$ is a constant that sets the overall scale. This form
captures the idea that the potential is driven by both the flow field
and the local entropy level. For small $S$ the potential is dominated by
the flow itself, while for larger $S$ the contribution of entropy
becomes important.

A more general family of couplings can be written as
\[
\Phi(E_f,S) = A_\Phi E_f \bigl(1 + S^\beta\bigr),
\]
with $\beta$ controlling the non-linear response to entropy. The present
baseline implementation uses the linear case $\beta = 1$.

% -------------------------------------------------------
\subsection{Expansion Rate \texorpdfstring{$H(E_f,S)$}{H(Ef,S)}}

EFC defines an effective expansion rate $H(E_f,S)$ as a function of the
energy-flow amplitude and the entropy level. In the simplest working
model we take
\begin{equation}
    H(E_f,S) = \sqrt{|E_f|}\, (1 + S).
\end{equation}

This choice reflects:

\begin{itemize}
    \item the dependence of expansion on the strength of the underlying
          energy flow, via $\sqrt{|E_f|}$;
    \item the modulation of this expansion by entropy, through $(1+S)$.
\end{itemize}

The functional form can be generalized, but this minimal ansatz gives a
clear and testable link between $E_f$, $S$ and large-scale dynamics. In
particular, late-time acceleration can be understood as arising from
changes in $S$ and $E_f$ rather than from an external dark-energy field.

% -------------------------------------------------------
\subsection{Rotation Curves and the Potential Gradient}

The effective circular velocity at radius $r$ in a stationary
configuration is given by
\[
v(r) = \sqrt{\, r \, \frac{\partial \Phi}{\partial r} \, }.
\]

Because $\Phi$ depends on both $E_f$ and $S$, and both of these are
shaped by the entropy landscape, the observed rotation curves are
emergent thermodynamic features. They do not require additional
non-baryonic matter components but arise from the combined roles of
grid resistance, energy flow and $c(S)$.

% -------------------------------------------------------
\subsection{Coupling to the Effective Light Speed}

The dynamical sector is not independent of light propagation. The
entropy-dependent effective light speed
\[
c(S) = c_0 \left( 1 + a_{\text{edge}}\, x(S)^2 \right)
\]
modifies how dynamical information is inferred from observations.

Even if the underlying $\Phi(E_f,S)$ and $E_f$ fields are fixed, the
measured velocities, time delays and inferred accelerations depend on
$c(S)$ along the relevant lines of sight. EFC-D therefore works together
with the light-propagation sector to produce the observed cosmological
signals.

% -------------------------------------------------------
\subsection{Summary}

EFC-D provides a thermodynamic definition of:

\begin{itemize}
    \item the energy-flow field $E_f$,
    \item the effective potential $\Phi(E_f,S)$,
    \item the expansion rate $H(E_f,S)$,
    \item and the coupling of these quantities to $c(S)$.
\end{itemize}

The resulting dynamics can be tested against large-scale structure,
expansion-history data and rotation-curve measurements within a single
unified framework.
