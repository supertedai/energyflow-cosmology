% =====================================================================
% Energy-Flow Cosmology — EFC-S Structural Profile Model (Master Spec)
% =====================================================================

\section*{EFC–S Structural Model: Formal Specification}

\subsection*{1. Purpose}

The EFC–S model defines the spatial structure of density, entropy, and
energy–flow profiles in Energy-Flow Cosmology (EFC). Unlike EFC–D, which
defines temporal evolution, the S model describes *static equilibrium*,
radial profiles, and spatial organisation of matter and entropy.

This model is used for predicting halo shapes, cluster profiles, lensing
behaviour and large-scale structural coherence.

\subsection*{2. Core Fields}

\begin{itemize}
    \item Entropy field: $S(\mathbf{x})$
    \item Structural density: $\rho(\mathbf{x})$
    \item Energy–flow potential: $E_f(\mathbf{x})$
    \item Grid resistance: $R(\mathbf{x})$
\end{itemize}

All functions are *time-independent* in the S-model.

\subsection*{3. Structural Equilibrium Condition}

The defining equation of the structural model is:

\[
\nabla E_f = 0.
\]

Given:

\[
E_f = \rho (1 - S),
\]

equilibrium implies:

\[
\rho(1 - S) = E_0,
\]

where $E_0$ is a spatial constant (locally defined for each structure).

This is the **EFC Structural Equilibrium Law**.

\subsection*{4. Structural Profile Equation}

Solving for density:

\[
\rho(\mathbf{x}) = \frac{E_0}{1 - S(\mathbf{x})}.
\]

Interpretation:

- Regions with lower entropy hold higher structure.
- As entropy increases outward, density decreases naturally.
- This produces smooth halo-like radial profiles without free parameters.

\subsection*{5. Entropy Profile Constraint}

EFC–S assumes gently increasing entropy with radius:

\[
\frac{dS}{dr} > 0,
\quad
| \frac{dS}{dr} | \ll 1.
\]

This produces:

- flat cores,  
- slowly declining mid-radii,  
- asymptotic tails.

\subsection*{6. Resistance Profile}

Resistance increases with entropy:

\[
R(r) = R(S(r)).
\]

Thus outer regions have higher propagation resistance.

This reproduces:

- gravitational-like effects,
- lensing magnification,
- dynamical slowing consistent with effective “mass”.

\subsection*{7. Combined Structural System}

The full S-model:

\[
\begin{aligned}
E_f &= \rho(1-S), \\
\nabla E_f &= 0, \\
\rho(r) &= \frac{E_0}{1 - S(r)}, \\
R &= R(S), \\
c_{\text{eff}} &= (1 + R)^{-1}.
\end{aligned}
\]

This defines the spatial structure of EFC systems.

\subsection*{8. Relationship to Other EFC Models}

\begin{itemize}
    \item EFC–D: describes time evolution of $S$, $\rho$, $E_f$, $R$.
    \item EFC–H: describes structure formation and halo boundaries.
    \item EFC–C$_0$: describes informational and cognitive resonance.
\end{itemize}

EFC–S provides the *static backbone* that all other models rely on.

