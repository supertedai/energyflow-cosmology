\section{EFC-S: Entropy Field and Structural Sector}

The structural sector (EFC-S) defines the entropy field $S(\mathbf{x})$
and its gradient. This field encodes local organization, anisotropy,
and structural complexity. It acts as a non-mass-based generator of
effective curvature and guides the flow of energy in the EFC framework.

\subsection{Entropy Field}

The entropy field is a scalar function over spatial configuration space:
\begin{equation}
    S : \mathbb{R}^3 \rightarrow [S_{\min}, S_{\max}].
\end{equation}

It represents the local degree of information organization.  
Low values correspond to homogeneous, featureless regions;  
high values correspond to strongly structured zones.

For practical computations we treat $S$ as differentiable and bounded.

\subsection{Entropy Gradient}

The gradient of the entropy field is:
\begin{equation}
    \nabla S(\mathbf{x}) =
    \left(
      \frac{\partial S}{\partial x},
      \frac{\partial S}{\partial y},
      \frac{\partial S}{\partial z}
    \right).
\end{equation}

The gradient quantifies directional changes in structural organization.
It provides the preferred direction for energy redistribution.

\subsection{Scaled Structural Gradient}

In the physical model, the raw gradient is scaled by a structural
coupling constant $k_S$:
\begin{equation}
    \mathbf{g}_S(\mathbf{x}) = k_S \nabla S(\mathbf{x}).
\end{equation}

This scaled gradient is the term that directly enters:

\begin{itemize}
    \item the energy-flow potential $V_{\text{EFC}}$,
    \item the expansion mapping $H(E_f, S)$,
    \item the structural force balance in EFC-S.
\end{itemize}

\subsection{Radial Reduction (Spherical Symmetry)}

For many astrophysical systems we approximate spherical symmetry:
\begin{equation}
    S(\mathbf{x}) = S(r),
\end{equation}
leading to
\begin{equation}
    \nabla S = \frac{dS}{dr} \hat{\mathbf{r}}.
\end{equation}

The structural gradient becomes:
\begin{equation}
    g_S(r) = k_S \frac{dS}{dr}.
\end{equation}

This form is used in the halo model (EFC-S) and the dynamical model
(EFC-D) for rotation curves.

\subsection{Interpretation in EFC}

The gradient acts as a geometric driver:

\begin{itemize}
    \item it introduces anisotropic structure,
    \item it modifies local effective curvature,
    \item it couples to the energy-flow field,
    \item it regulates stability and shape of halos.
\end{itemize}

EFC-S is therefore the structural backbone of the full EFC model,
and all dynamical sectors reference $g_S$ directly.
