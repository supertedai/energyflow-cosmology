\section{EFC-S: Structural Sector}

EFC-S describes how structures form and persist within the entropy
field $S$ and the energy-flow potential $\Phi(E_f,S)$. Structure arises
from thermodynamic balance between focusing/defocusing tendencies and
the propagation properties encoded in $c(S)$.

\subsection{Entropy and Structural Regimes}

The entropy endpoints
\[
S_0, \qquad S_1,
\]
mark low-entropy focusing and high-entropy defocusing regimes. The
mid-entropy value
\[
S_{\text{mid}} = \frac{1}{2}(S_0 + S_1)
\]
defines a natural transition zone.

\subsection{Entropy-Driven Stability Band}

Because the effective light speed $c(S)$ reaches its minimum near the
mid-entropy level $S_{\text{mid}}$, propagation slows in this region:

\[
c_{\text{min}} = c(S_{\text{mid}}).
\]

This creates a natural \emph{stability band} in which energy-flow
structures form persistent halo-like configurations. Reduced
propagation speed increases effective confinement and stabilizes the
rotation-curve plateau without requiring non-baryonic dark matter.

\subsection{Coupling to the Potential}

Structure formation reflects a balance among:
\[
E_f, \qquad \Phi(E_f,S), \qquad c(S).
\]

The potential
\[
\Phi(E_f,S) = A_\Phi E_f (1 + S)
\]
together with the propagation law $c(S)$ reproduces observed galactic
structural features as emergent thermodynamic behavior.
