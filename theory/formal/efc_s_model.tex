\section{EFC-S: Entropy-Stabilised Structures}

The structural sector of EFC describes how matter, light, and energy
organize themselves in regions with non-trivial entropy gradients. The
presence of the entropy-dependent effective speed of light,
\[
c(S) = c_0 \left( 1 + a_{\text{edge}} \, x(S)^2 \right),
\]
plays a central role in determining both stability conditions and
observable properties of halo-like configurations.

\subsection{Entropy Gradients and Structural Support}

Structures in EFC form where entropy gradients guide the energy-flow
field $E_f(\mathbf{x})$. In such regions, the grid exhibits varying
resistance, which directly modulates the propagation speed of light.

\paragraph{Focusing regime (s$_0$-side).}

Near s$_0$ (low entropy, high structure), we have $x(S)\to -1$ and
$c(S)$ is enhanced. Enhanced propagation speed leads to inward focusing
of light trajectories, which increases effective surface brightness and
reduces observed time delays. This focusing contributes naturally to
the appearance of halo-like rotation curves without invoking additional
mass components.

\paragraph{Defocusing regime (s$_1$-side).}

Near s$_1$ (high entropy, structural loosening), we have $x(S)\to +1$
and $c(S)$ increases again. Light traverses these regions more rapidly,
leading to outward defocusing. This contributes to large-scale
acceleration effects and modified anisotropies without a separate
dark-energy component.

\paragraph{Mid-entropy stability band.}

In the mid-entropy region $S \approx S_{\text{mid}}$, $c(S)$ is at its
minimum. This region exhibits maximal grid resistance and supports
long-lived structural configurations. Many halo-like formations arise
naturally in this band.

\subsection{Structural Radius and Observables}

Because focusing, defocusing, and stability depend on the functional
form of $c(S)$, even identical mass distributions can produce different
observable signatures depending on the entropy field.

Observable consequences include:

\begin{itemize}
    \item modified rotation curves from the interplay of $\Phi(E_f,S)$
          and the focusing contribution of $c(S)$;
    \item altered lensing magnification profiles in regions where the
          entropy approaches s$_0$ or s$_1$;
    \item anisotropic surface–brightness patterns arising from spatial
          variations in $c(S)$;
    \item structural stability conditions governed jointly by the entropy
          gradient and the mid-entropy minimum of $c(S)$.
\end{itemize}

EFC-S thus provides a unified description of halo-like phenomena in terms
of entropy structure, energy flow, and the emergent propagation law
$c(S)$, without requiring additional matter components.
