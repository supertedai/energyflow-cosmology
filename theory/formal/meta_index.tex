% ============================================================
% META SECTIONS — COLLECTED
% ============================================================

\section{Meta-Reflective Theory Development}

Energy-Flow Cosmology is developed through an iterative, reflective
method where thermodynamic structure, empirical constraints, and
mathematical consistency co-evolve. Adjustments to $S(\mathbf{x})$,
$E_f$, $c(S)$ and $H(E_f,S)$ are made as new empirical results emerge.

\subsection*{Method Layers}
\begin{itemize}
    \item \textbf{Observation Layer:} SPARC, DESI, JWST, lensing systems.
    \item \textbf{Mathematical Layer:} Thermodynamic potentials and
          entropy-driven propagation.
    \item \textbf{Interpretation Layer:} Structural stability,
          quasi-equilibria, entropy endpoints.
\end{itemize}

\newpage

\section{Meta-Architecture Specification}

The global architecture of EFC is defined by the interaction:
\[
S(\mathbf{x})
\Rightarrow E_f(\mathbf{x})
\Rightarrow \Phi(E_f,S)
\Rightarrow c(S)
\Rightarrow H(E_f,S)
\Rightarrow \text{Observables}.
\]

Stability conditions such as:
\[
\nabla E_f \approx 0
\quad\Rightarrow\quad \text{quasi-stable halo}
\]
and small entropy gradients encode structural persistence without
dark components.
