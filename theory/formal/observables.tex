\section{Observables in the EFC Framework}

In EFC, all observables based on null propagation depend on the
entropy-dependent effective light speed $c(S)$ rather than a universal
constant $c$. The propagation of light is therefore tied directly to the
entropy field structure sampled along a photon trajectory.

\subsection{Null Propagation in an Entropy Field}

For any photon trajectory $\gamma$, the observed travel time is
\[
t_{\text{obs}} = \int_{\gamma} \frac{dl}{c(S(l))},
\]
where $S(l)$ is the entropy encountered along the path. The effective
light speed is defined as
\[
c(S) = c_0 \left( 1 + a_{\text{edge}}\, x(S)^2 \right),
\qquad
x(S) = \frac{S - S_{\text{mid}}}{\Delta S / 2}.
\]

Because $c(S)$ varies with entropy, observables become environmental
rather than purely geometric quantities.

\subsection{Rotation Curves}

The observed rotation velocity depends jointly on
\[
\partial_r \Phi(E_f,S)
\quad \text{and} \quad
c(S(r)),
\]
leading to the modified observational mapping
\[
v_{\text{obs}}(r)
    = f\!\bigl(\partial_r \Phi,\, c(S(r))\bigr).
\]

The mid-entropy minimum of $c(S)$ produces enhanced confinement and
naturally yields flat rotation-curve plateaus without invoking
non-baryonic dark matter.

\subsection{Gravitational Lensing}

Time-delay ratios follow
\[
\Delta t_{ij}
  = \int_{\gamma_i} \frac{dl}{c(S(l))}
  - \int_{\gamma_j} \frac{dl}{c(S(l))}.
\]

Identical mass distributions may produce different lensing signatures
depending solely on the entropy landscape.

\subsection{Redshift--Distance Relations}

Because the light-propagation speed varies with $S$, luminosity distance
and angular-diameter distance acquire entropy-dependent corrections.
Late-time acceleration arises through $c(S)$ alone and does not require
a separate dark-energy field.
