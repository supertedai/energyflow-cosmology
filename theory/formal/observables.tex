% ============================================================
% EFC Observables
% Clean, Unicode-safe, PNG-free version
% ============================================================

\section{Observables in Energy-Flow Cosmology}

Energy-Flow Cosmology (EFC) predicts that all observable quantities are
modulated by the entropy field $S(\mathbf{x})$, the energy-flow field
$E_f(\mathbf{x})$, and the derived effective light speed $c(S)$.
This section summarizes how these quantities appear in observations.

% ------------------------------------------------------------
\subsection{Light Propagation}

The light-travel time along a photon trajectory $\gamma$ is
\[
t_{\text{obs}}
= \int_{\gamma} \frac{dl}{c(S(l))},
\]
where $c(S)$ is the entropy-dependent effective speed of light.

Because $c(S)$ varies across the grid:

\begin{itemize}
    \item regions near $s_0$ (low entropy, high structure) increase the local speed of light, producing focusing;
    \item regions near $s_1$ (high entropy, structural loosening) increase $c(S)$ at large scales, producing defocusing;
    \item mid-entropy regions produce the slowest propagation and the longest delays.
\end{itemize}

These effects modify classical interpretations of redshifts, distances,
time-delays in lensing systems, and luminosity-distance relations.

% ------------------------------------------------------------
\subsection{Lensing}

In GR, lensing depends only on the mass distribution.
In EFC, lensing depends on \emph{both} mass and entropy structure:

\[
\text{Lensing} = \text{Mass Distribution} + S(\mathbf{x}) + c(S).
\]

Consequently:

\begin{itemize}
    \item focusing from $s_0$ regions can mimic excess mass;
    \item defocusing from $s_1$ regions can mimic mass deficits;
    \item time-delay ratios between lens images become modified;
    \item apparent shear patterns depend on how photon paths sample $S(\mathbf{x})$.
\end{itemize}

% ------------------------------------------------------------
\subsection{Rotation Curves}

Circular velocities follow from the EFC potential:
\[
v(r) = \sqrt{\, r \, \frac{\partial \Phi}{\partial r} \, }.
\]

Because $\Phi(E_f,S)$ depends on both $E_f$ and $S$, the resulting
rotation curves can flatten without invoking dark matter.  
In particular:

\begin{itemize}
    \item stable regions around $S_{\text{mid}}$ produce broad bands of self-regulated velocities;
    \item transitions toward $s_0$ and $s_1$ produce characteristic slow variations;
    \item observational inferences of $v(r)$ must account for entropy-modulated $c(S)$.
\end{itemize}

% ------------------------------------------------------------
\subsection{Expansion Rate}

The EFC expansion rate is:
\[
H(E_f,S) = \sqrt{|E_f|}\,(1+S),
\]

which implies that cosmic acceleration arises from the joint evolution
of $(E_f,S)$ rather than from dark energy or a cosmological constant.

% ------------------------------------------------------------
\subsection{Diagram of Dependencies}

The core dependency chain for observables is:

\[
S(\mathbf{x})
\;\Longrightarrow\;
E_f(\mathbf{x})
\;\Longrightarrow\;
\Phi(E_f,S)
\;\Longrightarrow\;
c(S)
\;\Longrightarrow\;
\text{Observables}.
\]

This captures the full thermodynamic origin of structure formation, 
photon propagation, and cosmic expansion within EFC.

% ------------------------------------------------------------
% Removed figure calling the missing PNG file.
% If you wish to re-enable it later, use:
%
% \begin{figure}[h!]
% \centering
% \includegraphics[width=\textwidth]{../../output/EFC_vs_LCDM_plot.png}
% \caption{Comparison of EFC baseline and $\Lambda$CDM rotation curves.}
% \end{figure}
%
% ------------------------------------------------------------

