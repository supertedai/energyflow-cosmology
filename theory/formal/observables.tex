\subsection{Light travel times and lensing with $c(S)$}

In EFC, light propagation is governed by the entropy–dependent effective
speed of light
\[
c(S) = c_0 \left( 1 + a_{\text{edge}} \, x(S)^2 \right),
\]
where
\[
x(S) = \frac{S - S_{\text{mid}}}{\Delta S / 2},
\qquad
S_{\text{mid}} = \frac{1}{2}(S_0 + S_1),
\qquad
\Delta S = S_1 - S_0.
\]

Thus, photons do not propagate with a fixed constant speed, but with a
locally varying speed determined by the entropy field sampled along their
path. The observable travel time for a photon following a trajectory
$\gamma$ is therefore
\begin{equation}
    t_{\text{obs}} = \int_{\gamma} \frac{\mathrm{d}l}{c(S(l))}.
\end{equation}

\paragraph{Consequences.}

Relative to a constant-$c$ model, the following effects arise naturally:

\begin{itemize}
    \item \textbf{Enhanced propagation near s$_0$ and s$_1$.}  
          In regions where $S$ approaches either endpoint (low-entropy
          s$_0$ or high-entropy s$_1$), $c(S)$ increases. Photons traverse
          these regions more quickly, reducing or reshaping time delays.
    
    \item \textbf{Modified lensing without additional mass.}  
          Because the propagation speed is a function of the entropy field
          rather than a universal constant, magnification patterns,
          distortion angles, and time-delay ratios can differ from GR-based
          predictions even when the underlying mass distribution is the same.
          These deviations arise from variations in $c(S)$ rather than
          from unseen mass components.
    
    \item \textbf{Redshift--distance relations depend on $c(S)$.}  
          Cosmological observables inferred from lightlike probes must be
          interpreted using the integral
          \[
              d_L(z),\, d_A(z) \propto \int \frac{\mathrm{d}l}{c(S(l))},
          \]
          not the usual expression with constant $c$.  
          This provides an alternative route to late-time acceleration
          without introducing a separate dark-energy component.
    
    \item \textbf{Path dependence becomes physical.}  
          Light rays passing through different entropy structures sample
          different functional forms of $c(S)$, making anisotropies and
          direction-dependent time delays a natural prediction of EFC.
\end{itemize}

In summary, lensing, time-delay measurements, and distance–redshift
relations in EFC are shaped not only by the mass-energy distribution but
also by the entropy field through the function $c(S)$.
