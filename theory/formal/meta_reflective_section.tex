% ===========================================================
% Meta-Reflective Theory Development
% ===========================================================

\section*{Meta-Reflective Theory Development}

The development of Energy-Flow Cosmology (EFC) follows an iterative,
meta-reflective process rather than a linear derivation.  
The core principles of the reflective loop are:

\begin{itemize}
    \item continuous refinement of equations through empirical constraints;
    \item reciprocal updating between entropy structure and expansion dynamics;
    \item use of conceptual invariants across thermodynamic layers;
    \item explicit tracking of assumptions and transformations.
\end{itemize}

This reflective structure allows the theory to evolve as a
self-updating system, where each layer informs the others:

\[
(E_f, S, c(S), H) \;\Longleftrightarrow\; \text{Observables}.
\]

% -----------------------------------------------------------
\subsection*{Reflective Loop Structure}

The reflective loop is defined by three components:

\begin{enumerate}
    \item \textbf{Observation layer:}  
    empirical data (SPARC, JWST, DESI, lensing).

    \item \textbf{Mathematical layer:}  
    thermodynamic equations connecting entropy and flow.

    \item \textbf{Interpretation layer:}  
    model adjustments, conceptual updates, and stability checks.
\end{enumerate}

The loop continues until the mapping between
thermodynamics and cosmological observables stabilizes.

% -----------------------------------------------------------
\subsection*{Role of Entropy}

Entropy acts as the organizing variable in all reflective steps:

\begin{itemize}
    \item setting structural boundaries;
    \item defining light-speed modulation;
    \item governing large-scale expansion;
    \item encoding the transition between $s_0$ and $s_1$.
\end{itemize}

This leads to a thermodynamic synthesis where
cosmic structure is the emergent result of gradients,
not additional hypothetical substances.

% -----------------------------------------------------------
\subsection*{Summary}

The meta-reflective approach ensures that the theory remains:

\begin{itemize}
    \item internally consistent,
    \item observationally grounded,
    \item thermodynamically constrained,
    \item adaptable to new data releases.
\end{itemize}

This section documents the methodological backbone of EFC's evolution.
