\section{EFC-D: Dynamical Sector}

The dynamical sector EFC-D describes how the energy-flow field $E_f$,
the effective potential $\Phi(E_f,S)$ and the expansion rate $H$
emerge from the entropy field $S$ and its gradients. It forms the
bridge between thermodynamics and the observable evolution of
structures.

% -------------------------------------------------------
\subsection{Energy-Flow Field}

We define the scalar energy-flow field
\[
E_f : \mathbb{R}^3 \rightarrow \mathbb{R},
\]
which encodes the strength and directionality of energy transport along
entropy gradients. In regions with strong gradients $\nabla S$, the
magnitude $|E_f|$ is large, while in nearly uniform regions $E_f$
approaches zero.

To leading order the flow aligns with the negative gradient of entropy,
\[
E_f \propto - \nabla S,
\]
reflecting the tendency of energy to organize along entropy-driven
channels.

% -------------------------------------------------------
\subsection{Effective Potential \texorpdfstring{$\Phi(E_f,S)$}{Phi(Ef,S)}}

The effective potential is defined as
\[
\Phi(E_f,S) = A_\Phi \, E_f \, (1 + S),
\]
where $A_\Phi$ is a constant setting the amplitude. This baseline form
captures how the interaction between energy flow and the entropy level
builds the potential landscape.

A more general non-linear family is given by
\[
\Phi(E_f,S) = A_\Phi E_f \bigl(1 + S^\beta\bigr),
\]
where $\beta$ modulates the strength of the coupling. The baseline
implementation adopts the linear case $\beta = 1$.

% -------------------------------------------------------
\subsection{Expansion Rate \texorpdfstring{$H(E_f,S)$}{H(Ef,S)}}

EFC defines an effective expansion rate as a derived quantity:
\begin{equation}
    H(E_f,S) = \sqrt{|E_f|}\, (1 + S).
\end{equation}

This functional form reflects two contributions:

\begin{itemize}
    \item $\sqrt{|E_f|}$ captures how the magnitude of energy flow sets
          the dynamical scale.
    \item $(1+S)$ represents the thermodynamic modulation of expansion.
\end{itemize}

In this formulation, late-time acceleration can arise naturally from
changes in $(E_f, S)$ without introducing an external dark-energy
component.

% -------------------------------------------------------
\subsection{Rotation Curves and the Potential Gradient}

For stationary configurations, the effective circular velocity at
radius $r$ follows from the potential gradient:
\[
v(r) = \sqrt{\, r \, \frac{\partial \Phi}{\partial r} \, }.
\]

Because $\Phi$ depends on both $E_f$ and $S$, and these are shaped
by the entropy landscape, the resulting rotation curves arise as
thermodynamic signatures. They do not require additional dark-matter
components but emerge from the coupled roles of grid resistance,
energy flow and the entropy field.

% -------------------------------------------------------
\subsection{Coupling to the Effective Light Speed}

The dynamical sector is intrinsically linked to the entropy-dependent
effective light speed
\[
c(S) = c_0 \left( 1 + a_{\text{edge}} \, x(S)^2 \right).
\]

Even with fixed underlying fields $E_f$ and $\Phi$, the inferred
dynamical quantities depend on how light samples $c(S)$ along
lines of sight. Observed velocities, accelerations, lensing
time delays, and distance measures are all modified through this
coupling.

Thus EFC-D and the light-propagation sector form a joint dynamical
system rather than independent modules.

% -------------------------------------------------------
\subsection{Summary}

EFC-D provides thermodynamic definitions of:

\begin{itemize}
    \item the energy-flow field $E_f$,
    \item the effective potential $\Phi(E_f,S)$,
    \item the expansion rate $H(E_f,S)$,
    \item and the coupling of these quantities to the entropy-dependent
          light speed $c(S)$.
\end{itemize}

Together these quantities generate large-scale structure evolution,
acceleration, and rotation-curve behaviour in a unified framework.
