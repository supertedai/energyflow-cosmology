% ============================================================
% EFC-C₀: Entropy-Dependent Light Propagation
% ============================================================

\section{EFC-C\textsubscript{0}: Entropy-Dependent Light Propagation}
\label{sec:efc_c0}

Energy-Flow Cosmology treats the effective speed of light as an
emergent property of the thermodynamic state of the grid. The
field $S(\mathbf{x})$ does not only modulate structure and
energy-flow, but also the propagation of information.

EFC-C\textsubscript{0} closes the chain
\[
  S(\mathbf{x})
  \;\Rightarrow\;
  E_f(\mathbf{x})
  \;\Rightarrow\;
  \Phi(E_f,S)
  \;\Rightarrow\;
  c(S)
  \;\Rightarrow\;
  H(E_f,S)
  \;\Rightarrow\;
  \text{Observables}.
\]

% ------------------------------------------------------------
\subsection{Entropy Endpoints and Normalised Coordinate}

We define two entropy endpoints
\begin{itemize}
  \item $s_0$: low-entropy, high-structure endpoint,
  \item $s_1$: high-entropy, high-diffusion endpoint,
\end{itemize}
with midpoint and span
\[
  S_{\text{mid}} = \frac{1}{2}(s_0 + s_1),
  \qquad
  \Delta S = s_1 - s_0.
\]

A normalised entropy coordinate is
\[
  x(S) = \frac{S - S_{\text{mid}}}{\Delta S/2},
\]
so that $x = -1$ corresponds to $S = s_0$ and $x = +1$ to $S = s_1$.

% ------------------------------------------------------------
\subsection{Effective Speed of Light and Refractive Index}

The effective speed of light is modeled as an even function around
$S_{\text{mid}}$:
\[
  c(S) = c_0 \left( 1 + a_{\text{edge}}\, x(S)^2 \right),
\]
where $a_{\text{edge}}$ controls the strength of edge-driven
modulation. The corresponding refractive index is
\[
  n(S) = \frac{c_0}{c(S)}
       = \frac{1}{1 + a_{\text{edge}}\, x(S)^2}.
\]

In the EFC grid frame, $c_0$ is the reference value in a
weakly-structured, mid-entropy background. Deviations are entirely
encoded in $S$.

% ------------------------------------------------------------
\subsection{Light Travel Time and Fermat Principle}

For a photon trajectory $\gamma$ the observed travel time is
\[
  t_{\text{obs}}
  = \int_{\gamma} \frac{dl}{c(S(l))}
  = \frac{1}{c_0} \int_{\gamma} n\bigl(S(l)\bigr)\, dl.
\]

The path is determined by a thermodynamic Fermat principle:
\[
  \delta \int_{\gamma} n\bigl(S(\mathbf{x})\bigr)\, dl = 0.
\]

Gradients in $S$ therefore act as an effective refractive medium,
even at fixed mass distribution.

% ------------------------------------------------------------
\subsection{Dual Focusing–Defocusing Behaviour}

The entropy endpoints have distinct optical roles:
\begin{itemize}
  \item Near $s_0$ (low entropy, strong structure) the local
    modulation of $c(S)$ and $n(S)$ produces focusing similar to
    mass overdensities.
  \item Near $s_1$ (high entropy, loosened structure) the same
    mechanism generates large-scale defocusing.
  \item Around $S_{\text{mid}}$ the propagation is slowest, giving
    maximal time delays.
\end{itemize}

This yields a dual-lens profile:
\begin{align*}
  s_0 &\;\Rightarrow\; \text{sharp, local focusing}, \\
  s_1 &\;\Rightarrow\; \text{broad, diffuse defocusing}.
\end{align*}

% ------------------------------------------------------------
\subsection{Coupling to the EFC Potential and Expansion}

Because $S(\mathbf{x})$ also controls the energy-flow potential
\(
  E_f = \rho(1-S)
\),
the same field that modulates $c(S)$ drives the potential
$\Phi(E_f,S)$ and the expansion rate
\[
  H(E_f,S) = \sqrt{|E_f|}\,(1+S).
\]

In EFC-C\textsubscript{0}, apparent anomalies in
\begin{itemize}
  \item lensing strength,
  \item time-delay ratios,
  \item inferred distances and expansion rates,
\end{itemize}
are interpreted as consequences of entropy-structured
light propagation rather than missing mass or a separate dark energy
component.
