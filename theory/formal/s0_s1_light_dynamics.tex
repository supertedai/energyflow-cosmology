% ===========================================================
% s0--s1 Light Propagation Dynamics
% Clean, self-contained LaTeX version
% ===========================================================

\section*{s0--s1 Light Propagation Dynamics}

This section describes how the entropy field determines
variations in the effective speed of light, photon travel time,
and lensing behavior across the EFC grid.

% -----------------------------------------------------------
\subsection*{Entropy Endpoints}

We define two entropy endpoints:

\begin{itemize}
    \item $s_0$: low-entropy, high-structure endpoint
    \item $s_1$: high-entropy, high-diffusion endpoint
\end{itemize}

The midpoint is

\[
S_{\text{mid}} = \frac{1}{2}(s_0 + s_1),
\]

and the entropy span is

\[
\Delta S = s_1 - s_0.
\]

We define the normalized entropy coordinate

\[
x(S) = \frac{S - S_{\text{mid}}}{\Delta S/2}.
\]

% -----------------------------------------------------------
\subsection*{Variation of the Effective Speed of Light}

The effective speed of light becomes:

\[
c(S) = c_0 \left( 1 + a_{\text{edge}} \, x(S)^2 \right),
\]

where $a_{\text{edge}}$ controls the strength of edge-driven variation.

Interpretation:

\begin{itemize}
    \item near $s_0$: locally increased $c(S)$ → focusing
    \item near $s_1$: large-scale increase in $c(S)$ → defocusing
    \item around $S_{\text{mid}}$: the slowest propagation regime
\end{itemize}

This yields a nonuniform propagation landscape across the cosmic grid.

% -----------------------------------------------------------
\subsection*{Light Travel Time}

A photon's travel time along a trajectory $\gamma$ is

\[
t_{\text{obs}} = \int_{\gamma} \frac{dl}{c(S(l))}.
\]

Regions of slow light $(c(S)\text{ small})$ increase observed delays,
while fast-light zones near $s_0$ and $s_1$ reduce them.

% -----------------------------------------------------------
\subsection*{Lensing Behavior}

Low-entropy endpoints act as focusing wells;  
high-entropy endpoints contribute to large-scale defocusing.

This yields a dual-lens structure:

\begin{itemize}
    \item $s_0$ regions produce sharp focusing, similar to mass overdensities,
    \item $s_1$ regions create gentle, large-area defocusing,
    \item transitions between the two define asymmetric lensing profiles.
\end{itemize}

Such behavior is consistent with halo asymmetries observed in
galaxy rotation curves and gravitational lensing systems.

% -----------------------------------------------------------
\subsection*{Summary}

The $s_0$--$s_1$ entropy axis determines:

\begin{itemize}
    \item effective $c(S)$,
    \item photon geodesics,
    \item observed delays,
    \item lensing,
    \item halo asymmetry,
    \item structural formation boundaries.
\end{itemize}

This provides a thermodynamic explanation for several phenomena
traditionally attributed to dark matter or modified gravity.
