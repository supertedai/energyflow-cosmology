\documentclass[11pt]{article}

% --------------------------------------------------
% Basic packages (safe for Pandoc → HTML + PDF)
% --------------------------------------------------
\usepackage[utf8]{inputenc}
\usepackage{amsmath, amssymb}
\usepackage{graphicx}
\usepackage{geometry}
\geometry{margin=2.4cm}

\usepackage{hyperref}
\hypersetup{
    colorlinks=true,
    linkcolor=blue,
    urlcolor=blue
}

% --------------------------------------------------
% Title metadata
% --------------------------------------------------
\title{Energy-Flow Cosmology --- Master Formal Specification}
\author{Morten Magnusson}
\date{\today}

\begin{document}

\maketitle

% -------------------
% CONTENT START
% -------------------

\section{EFC-D: Energy-Flow Dynamics}

The local energy-flow potential \(E_f\) is defined as a function of
mass density \(\rho\) and entropy \(S\):

\[
E_f = \rho (1 - S).
\]

Interpretation:

\begin{itemize}
  \item high density and low entropy \( \Rightarrow \) large \(E_f\)
  \item high entropy suppresses \(E_f\)
\end{itemize}

\subsection{Density}

The mass density is:

\[
\rho = \frac{m}{V}.
\]

\subsection{Energy-Flow Rate}

The temporal change is:

\[
\frac{dE_f}{dt} = \nabla_t E_f.
\]


% --------------------------------------------------
\section{EFC-S: Entropy Field and Gradient}

The entropy field \(S(\mathbf{x})\) ranges between ordered and disordered regions:

\[
S(\mathbf{x}) \in [0,1].
\]

Entropy gradient:

\[
\nabla S(\mathbf{x}) =
\left(
\frac{\partial S}{\partial x},
\frac{\partial S}{\partial y},
\frac{\partial S}{\partial z}
\right).
\]

Magnitude:

\[
|\nabla S| =
\sqrt{
\left(\frac{\partial S}{\partial x}\right)^2 +
\left(\frac{\partial S}{\partial y}\right)^2 +
\left(\frac{\partial S}{\partial z}\right)^2
}.
\]


% --------------------------------------------------
\section{s0 → s1 Light Propagation}

Entropy endpoints:

\[
S_{\text{mid}} = \frac{1}{2}(s_0 + s_1),
\qquad
\Delta S = s_1 - s_0.
\]

Normalized entropy coordinate:

\[
x(S) = \frac{S - S_{\text{mid}}}{\Delta S / 2}.
\]

Effective light speed:

\[
c(S) = c_0\left(1 + a_{\text{edge}}\, x(S)^2\right).
\]

Light-travel time:

\[
t_{\text{obs}} = \int_{\gamma} \frac{dl}{c(S(l))}.
\]


% --------------------------------------------------
\section{Observables}

\subsection{Lensing}

\[
\text{Lensing} =
\text{Mass Distribution}
+ S(\mathbf{x})
+ c(S).
\]

\subsection{Rotation Curves}

\[
v(r) = \sqrt{r \, \frac{\partial \Phi}{\partial r}}.
\]

\subsection{Expansion Rate}

\[
H(E_f, S) = \sqrt{|E_f|}(1 + S).
\]

Core chain:

\[
S(\mathbf{x})
\rightarrow
E_f(\mathbf{x})
\rightarrow
\Phi(E_f, S)
\rightarrow
c(S)
\rightarrow
H(E_f, S)
\rightarrow
\text{Observables}.
\]


% --------------------------------------------------
\section{Figure}

\begin{figure}[h!]
\centering
\includegraphics[width=0.95\linewidth]{EFC_vs_LCDM_plot.png}
\caption{Comparison between EFC baseline behaviour and \(\Lambda\)CDM.}
\end{figure}


% --------------------------------------------------
\section{Meta-Reflective Development}

The theory evolves through iterative refinement using:

\begin{itemize}
    \item SPARC rotation curves
    \item JWST early galaxies
    \item DESI expansion data
    \item gravitational lensing structure
\end{itemize}

\end{document}
