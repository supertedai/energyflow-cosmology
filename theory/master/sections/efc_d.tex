\subsection*{Overview}

EFC-D beskriver dynamikken i et kosmologisk system som et resultat av
energiflyt langs entropigradienter. Modellen tar utgangspunkt i et
energifelt $E_f$ som varierer med entropitetthet $S$.

\subsection*{Core Relations}

Grunnideen er at dynamikk oppstår når:

\[
\nabla S \neq 0
\]

Energi beveger seg langs stigende entropi:

\[
\vec{v}_E = -k \, \nabla S
\]

Der konstanten $k$ styrer responsen i feltet.

\subsection*{Effective Potential}

Vi definerer et effektivt potensial for energiflyt:

\[
\Phi_E(S) = -\int E_f(S) \, dS
\]

som styrer akselerasjonen:

\[
\vec{a} = -\nabla \Phi_E
\]

\subsection*{Expansion Rate}

Utvidelsesraten er gitt ved:

\[
H(S) = \sqrt{|E_f(S)|} \, \left(1 + S \right)
\]

Dette brukes videre i strukturlagene og observasjonsmodulene.
