\documentclass[11pt,a4paper]{article}

\usepackage[utf8]{inputenc}
\usepackage[T1]{fontenc}
\usepackage{lmodern}
\usepackage[english]{babel}
\usepackage{amsmath, amssymb, amsfonts}
\usepackage{geometry}
\usepackage{graphicx}
\usepackage{hyperref}
\usepackage{microtype}
\usepackage{bm}

\geometry{margin=2.5cm}

\hypersetup{
  colorlinks=true,
  linkcolor=blue,
  urlcolor=blue,
  citecolor=blue,
  pdfauthor={M. Magnusson},
  pdftitle={Energy-Flow Cosmology (EFC) --- Master Specification v1.0}
}

\title{Energy-Flow Cosmology (EFC)\\[0.25em]
  \large Master Specification v1.0}
\author{M. Magnusson}
\date{\today}

\begin{document}

\maketitle

\begin{abstract}
Energy-Flow Cosmology (EFC) treats the universe as a thermodynamic information system driven by
gradients in energy flow and entropy. Instead of introducing invisible matter or energy components,
EFC starts from energy distribution, entropy gradients and information capacity. The theory is
organised into three tightly coupled base layers:

\begin{itemize}
  \item EFC-S: structural and halo-level descriptions.
  \item EFC-D: energy-flow dynamics on top of these structures.
  \item EFC-C$_0$: base mapping between entropy and information capacity.
\end{itemize}

This document fixes notation and baseline equations for these three layers, and provides a compact,
mathematically explicit core that higher-level models, simulations and epistemic layers can reference
without ambiguity. The figures included here are schematic and illustrate the theoretical structure
rather than final data-calibrated fits.
\end{abstract}

\tableofcontents

\section{Frontmatter}

This document is the canonical master specification for Energy-Flow Cosmology (EFC). It defines the
formal structure and relations between:

\begin{itemize}
  \item \textbf{EFC-S}: structural and halo-level descriptions,
  \item \textbf{EFC-D}: energy-flow dynamics on top of these structures,
  \item \textbf{EFC-C$_0$}: base mapping between entropy and information capacity.
\end{itemize}

The goal is a compact, mathematically explicit core that higher-level models, simulations and
epistemic layers can reference without ambiguity.

\section{Overview}

EFC treats the universe as a thermodynamic information system driven by gradients in energy flow and
entropy. Instead of adding invisible components, the model starts from:

\begin{itemize}
  \item energy distribution,
  \item entropy gradients,
  \item information capacity.
\end{itemize}

The three base layers are:

\begin{itemize}
  \item \textbf{EFC-S} defines how low-entropy matter distributions organise into halo-like structures;
  \item \textbf{EFC-D} defines how local energy-flow potentials and their gradients shape dynamics;
  \item \textbf{EFC-C$_0$} defines how entropy and structure map to information capacity and cognitive potential.
\end{itemize}

A central object is the local energy-flow potential
\begin{equation}
  E_f(\bm{x}) = \rho(\bm{x})\bigl(1 - S(\bm{x})\bigr),
\end{equation}
which couples density and entropy into a single field.

\section{Illustrative Field and Profiles}

\subsection{Energy-flow potential field $E_f(\rho,S)$}

Figure~\ref{fig:Ef-map} shows a schematic map of the energy-flow potential
$E_f(\rho,S) = \rho(1-S)$ over the $(\rho,S)$ plane.

\begin{figure}[h]
  \centering
  \includegraphics[width=0.65\textwidth]{efc_Ef_rho_S_heatmap.png}
  \caption{Schematic heatmap of the energy-flow potential $E_f(\rho,S) = \rho(1-S)$.}
  \label{fig:Ef-map}
\end{figure}

\subsection{Halo profiles: mass and entropy}

\begin{figure}[h]
  \centering
  \includegraphics[width=0.65\textwidth]{efc_halo_rho_profile.png}
  \caption{Schematic halo mass density profile $\rho_h(r)$.}
  \label{fig:halo-rho}
\end{figure}

\begin{figure}[h]
  \centering
  \includegraphics[width=0.65\textwidth]{efc_halo_S_profile.png}
  \caption{Schematic halo entropy profile $S_h(r)$.}
  \label{fig:halo-S}
\end{figure}

\subsection{Rotation curves and projected density}

\begin{figure}[h]
  \centering
  \includegraphics[width=0.65\textwidth]{efc_schematic_rotation_curves.png}
  \caption{Schematic rotation curves: EFC-like vs NFW-like.}
  \label{fig:rotcurves}
\end{figure}

\begin{figure}[h]
  \centering
  \includegraphics[width=0.65\textwidth]{efc_schematic_lensing_profile.png}
  \caption{Schematic projected surface density $\Sigma(R)$.}
  \label{fig:lensing}
\end{figure}

\subsection{Expansion history and information capacity}

\begin{figure}[h]
  \centering
  \includegraphics[width=0.65\textwidth]{efc_schematic_Hz.png}
  \caption{Schematic expansion history $H(z)/H_0$.}
  \label{fig:Hz}
\end{figure}

\begin{figure}[h]
  \centering
  \includegraphics[width=0.65\textwidth]{efc_information_capacity_vs_S.png}
  \caption{Information capacity $I(S) \propto (1-S)$ at fixed $\rho$.}
  \label{fig:info}
\end{figure}

\clearpage

\section{Part I --- EFC-S: Structure / Halo Layer}

\subsection{S0. Low-entropy anchors}

Let $s(\bm{x})$ denote the entropy density. A low-entropy anchor is a region $\mathcal{A}$ such that
\begin{equation}
  \langle s \rangle_{\mathcal{A}} \ll \langle s \rangle_{\text{background}}.
\end{equation}

\subsection{S1. Halo Model of Entropy}

\begin{equation}
  \rho_h(r), \qquad S_h(r).
\end{equation}

\subsection{S2. Radial profiles}

\begin{equation}
  \rho_h(r) = \rho_0 f\!\left(\frac{r}{r_s}\right), \qquad
  S_h(r) = S_0 + (1 - S_0) g\!\left(\frac{r}{r_c}\right).
\end{equation}

\section{Part II --- EFC-D: Energy-Flow Dynamics}

\subsection{D0. Local energy-flow potential}

\begin{equation}
  E_f = \rho(1 - S).
\end{equation}

\subsection{D1. Rate}

\begin{equation}
  \frac{dE_f}{dt} = (1 - S)\frac{d\rho}{dt} - \rho\frac{dS}{dt}.
\end{equation}

\subsection{D2. Gradient}

\begin{equation}
  \nabla E_f = (1 - S)\nabla \rho - \rho\nabla S.
\end{equation}

\begin{equation}
  \bm{a} \propto -\nabla E_f.
\end{equation}

\section{Part III --- EFC-C$_0$}

\subsection{C0. Information capacity}

\begin{equation}
  I \propto \rho(1-S).
\end{equation}

\subsection{C1. Load}

\begin{equation}
  I_{\text{tot}} = \int_{\mathcal{R}} I(\bm{x})\,dV, \qquad
  L = \frac{I_{\text{used}}}{I_{\text{tot}}}.
\end{equation}

\subsection{C2. Coupling}

\begin{equation}
  \frac{dI}{dt} = \alpha\frac{dE_f}{dt} - \beta D_I.
\end{equation}

\section*{How to Cite}

This document is the formal master specification for the Energy-Flow Cosmology (EFC) framework.

\noindent \textbf{Citation:}\\
Magnusson, M. (2025).
\textit{Energy-Flow Cosmology (EFC) — Master Specification v1.0}. Figshare.\\
\href{https://doi.org/10.6084/m9.figshare.30630500}{DOI: 10.6084/m9.figshare.30630500}

\noindent \textbf{Author:} M. Magnusson \\
\textbf{ORCID:} \href{https://orcid.org/0009-0002-4860-5095}{0009-0002-4860-5095}

\noindent \textbf{Licence:} CC-BY 4.0

\medskip

\noindent This PDF corresponds to the archived and versioned research object on Figshare.
Future versions (EFC v1.x, v2.x) will reference this DOI as the baseline formal specification.

\section{References}

This specification may be supplemented with external references in future versions.

\end{document}
