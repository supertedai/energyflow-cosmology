\documentclass[11pt,a4paper]{article}

\usepackage[utf8]{inputenc}
\usepackage[T1]{fontenc}
\usepackage{lmodern}
\usepackage[english]{babel}
\usepackage{amsmath, amssymb, amsfonts}
\usepackage{geometry}
\usepackage{graphicx}
\usepackage{hyperref}
\usepackage{microtype}
\usepackage{bm}
\usepackage{titlesec}
\usepackage{setspace}

\geometry{margin=2.5cm}
\onehalfspacing

\hypersetup{
  colorlinks=true,
  linkcolor=blue,
  urlcolor=blue,
  citecolor=blue,
  pdfauthor={M. Magnusson},
  pdftitle={Energy-Flow Cosmology (EFC) --- Master Specification v1.1}
}

\titleformat{\section}
  {\normalfont\large\bfseries}{\thesection}{1em}{}
\titleformat{\subsection}
  {\normalfont\normalsize\bfseries}{\thesubsection}{1em}{}

\title{Energy-Flow Cosmology (EFC)\\[0.25em]
  \large Master Specification v1.1}
\author{M. Magnusson}
\date{\today}

\begin{document}

\maketitle

\vspace{-0.5em}
\begin{flushright}
{\small DOI: \href{https://doi.org/10.6084/m9.figshare.30630500}{10.6084/m9.figshare.30630500}}
\end{flushright}
\vspace{0.5em}

\begin{abstract}
Energy-Flow Cosmology (EFC) treats the universe as a thermodynamic information system driven by
gradients in energy flow and entropy. Instead of introducing invisible matter or energy components,
EFC starts from energy distribution, entropy gradients and information capacity. The theory is
organised into three tightly coupled base layers:
\begin{itemize}
  \item EFC-S: structural and halo-level descriptions,
  \item EFC-D: energy-flow dynamics on top of these structures,
  \item EFC-C$_0$: base mapping between entropy and information capacity.
\end{itemize}
This document fixes notation and baseline equations for these three layers, and provides a compact,
mathematically explicit core that higher-level models, simulations and epistemic layers can reference
without ambiguity. The figures included here are schematic and illustrate the theoretical structure
rather than final data-calibrated fits.
\end{abstract}

\clearpage
\tableofcontents
\clearpage

\section{Frontmatter}

This document is the canonical master specification for Energy-Flow Cosmology (EFC). It defines the
formal structure and relations between:
\begin{itemize}
  \item \textbf{EFC-S}: structural and halo-level descriptions,
  \item \textbf{EFC-D}: energy-flow dynamics on top of these structures,
  \item \textbf{EFC-C$_0$}: base mapping between entropy and information capacity.
\end{itemize}
The goal is a compact, mathematically explicit core that higher-level models, simulations and
epistemic layers can reference without ambiguity.

\subsection*{Version history}

\begin{itemize}
  \item v1.0: Initial formal master specification (structure, fields, schematic figures).
  \item v1.1: Layout polish, improved spacing, figure placement, and explicit DOI frontmatter.
\end{itemize}

\clearpage
\section{Overview}

EFC treats the universe as a thermodynamic information system driven by gradients in energy flow and
entropy. Instead of adding invisible components, the model starts from:
\begin{itemize}
  \item energy distribution,
  \item entropy gradients,
  \item information capacity.
\end{itemize}
The three base layers are:
\begin{itemize}
  \item \textbf{EFC-S} defines how low-entropy matter distributions organise into halo-like structures,
  \item \textbf{EFC-D} defines how local energy-flow potentials and their gradients shape dynamics,
  \item \textbf{EFC-C$_0$} defines how entropy and structure map to information capacity and cognitive potential.
\end{itemize}
A central object is the local energy-flow potential
\begin{equation}
  E_f(\bm{x}) = \rho(\bm{x})\bigl(1 - S(\bm{x})\bigr),
\end{equation}
which couples density and entropy into a single field.

\clearpage
\section{Illustrative Field and Profiles}

This section collects schematic figures that visualise the basic EFC fields and profiles. They are
theoretical examples consistent with the definitions in the later sections.

\subsection{Energy-flow potential field $E_f(\rho,S)$}

Figure~\ref{fig:Ef-map} shows a schematic map of the energy-flow potential
$E_f(\rho,S) = \rho(1-S)$ over the $(\rho,S)$ plane.

\begin{figure}[htbp]
  \centering
  \includegraphics[width=0.7\textwidth]{efc_Ef_rho_S_heatmap.png}
  \caption{Schematic heatmap of the energy-flow potential $E_f(\rho,S) = \rho(1-S)$
  as a function of mass density $\rho$ and dimensionless entropy $S$.}
  \label{fig:Ef-map}
\end{figure}

\subsection{Halo profiles: mass and entropy}

EFC-S models halos as joint profiles in mass density and entropy, $\rho_h(r)$ and $S_h(r)$.

\begin{figure}[htbp]
  \centering
  \includegraphics[width=0.7\textwidth]{efc_halo_rho_profile.png}
  \caption{Schematic halo mass density profile $\rho_h(r)$.}
  \label{fig:halo-rho}
\end{figure}

\begin{figure}[htbp]
  \centering
  \includegraphics[width=0.7\textwidth]{efc_halo_S_profile.png}
  \caption{Schematic halo entropy profile $S_h(r)$.}
  \label{fig:halo-S}
\end{figure}

\subsection{Rotation curves and projected density}

Given a halo profile, EFC-D can be used to derive effective rotation curves and projected surface
densities. Figure~\ref{fig:rotcurves} shows a schematic comparison between an EFC-like rotation curve
and an NFW-like reference. Figure~\ref{fig:lensing} shows a corresponding schematic projected
surface-density profile.

\begin{figure}[htbp]
  \centering
  \includegraphics[width=0.7\textwidth]{efc_schematic_rotation_curves.png}
  \caption{Schematic rotation curves for an EFC-like halo compared to an NFW-like reference.}
  \label{fig:rotcurves}
\end{figure}

\begin{figure}[htbp]
  \centering
  \includegraphics[width=0.7\textwidth]{efc_schematic_lensing_profile.png}
  \caption{Schematic projected surface density $\Sigma(R)$ associated with an EFC-like halo profile.}
  \label{fig:lensing}
\end{figure}

\subsection{Expansion history and information capacity}

EFC treats the effective expansion rate $H(z)$ as a derived quantity from flow and entropy, rather
than a primary parameter. Figure~\ref{fig:Hz} shows a schematic comparison between an EFC-like
expansion history and a $\Lambda$CDM-like reference. Figure~\ref{fig:info} shows a simple
information-capacity curve $I(S) \propto (1-S)$ at fixed density, relevant for EFC-C$_0$.

\begin{figure}[htbp]
  \centering
  \includegraphics[width=0.7\textwidth]{efc_schematic_Hz.png}
  \caption{Schematic effective expansion history $H(z)/H_0$ for an EFC-like model compared to a
  $\Lambda$CDM-like reference.}
  \label{fig:Hz}
\end{figure}

\begin{figure}[htbp]
  \centering
  \includegraphics[width=0.7\textwidth]{efc_information_capacity_vs_S.png}
  \caption{Schematic information capacity $I(S) \propto (1-S)$ at fixed density, illustrating the
  EFC-C$_0$ baseline relation between entropy and information capacity.}
  \label{fig:info}
\end{figure}

\clearpage
\section{Part I --- EFC-S: Structure / Halo Layer}

\subsection{S0. Low-entropy anchors}

EFC-S starts from the idea that structure forms around low-entropy anchors. These are local regions
where matter and energy are concentrated in configurations that allow sustained energy flows.

Let $s(\bm{x})$ denote the entropy density at position $\bm{x}$. A low-entropy anchor is a region
$\mathcal{A}$ such that
\begin{equation}
  \langle s \rangle_{\mathcal{A}} \ll \langle s \rangle_{\text{background}},
\end{equation}
where $\langle s \rangle$ denotes a coarse-grained average. These anchors serve as seeds for halo
formation and long-range correlations in the energy-flow field.

\subsection{S1. Halo Model of Entropy}

In EFC-S, halos are not only mass overdensities, but also \emph{entropy-structured} regions. A halo
profile is described by both mass density and entropy:
\begin{equation}
  \rho_h(r), \qquad S_h(r),
\end{equation}
where $\rho_h(r)$ is the radial mass density profile and $S_h(r)$ is a dimensionless entropy profile
normalised to $[0,1]$. The combination $(\rho_h(r), S_h(r))$ defines how effective a halo is as a
driver for energy flows in EFC-D.

\subsection{S2. Radial profiles and halo classes}

EFC-S allows families of halos parameterised by a small set of structural parameters (for example
central density, scale radius and entropy core size). A simple example parametrisation is
\begin{equation}
  \rho_h(r) = \rho_0\, f\!\left(\frac{r}{r_s}\right),
  \qquad
  S_h(r) = S_0 + (1 - S_0)\, g\!\left(\frac{r}{r_c}\right),
\end{equation}
where $f$ and $g$ are chosen shape functions, $r_s$ is a mass scale radius, and $r_c$ is an entropy
core radius. Different functional choices represent different halo classes in the EFC-S catalogue.

\clearpage
\section{Part II --- EFC-D: Energy-Flow Dynamics}

\subsection{D0. Local energy-flow potential $E_f(\rho,S)$}

The local energy-flow potential $E_f$ depends on the mass density $\rho$ and entropy $S$. It captures
how much structured energy-flow capacity a region has. At the baseline level,
\begin{equation}
  E_f = \rho\,(1 - S).
\end{equation}
High density with low entropy yields large $E_f$. High entropy suppresses $E_f$ even for dense
regions.

\subsection{D0.2. Mass density}

Mass density is defined in the usual way:
\begin{equation}
  \rho = \frac{m}{V},
\end{equation}
where $m$ is mass in a local region and $V$ is the associated volume.

\subsection{D1. Energy-flow rate and temporal evolution}

The temporal change of the energy-flow potential defines an energy-flow rate:
\begin{equation}
  \frac{dE_f}{dt} = \nabla_t E_f,
\end{equation}
where $\nabla_t$ is the derivative along the chosen time parameter (cosmic time or another evolution
parameter).

Using the definition of $E_f$ and applying the product rule, one obtains
\begin{equation}
  \frac{dE_f}{dt}
    = (1 - S)\,\frac{d\rho}{dt}
    - \rho\,\frac{dS}{dt}.
\end{equation}
This separates contributions from density change and entropy change: a region can lose energy-flow
potential by losing mass, by gaining entropy, or by both.

\subsection{D2. Spatial gradients and effective acceleration}

Spatial gradients in $E_f$ define preferred directions of energy flow. At the field level,
\begin{equation}
  \nabla E_f(\bm{x})
    = (1 - S(\bm{x}))\,\nabla \rho(\bm{x})
    - \rho(\bm{x})\,\nabla S(\bm{x}),
\end{equation}
which follows directly from the definition via the product rule.

At the level of an effective description, one can introduce an acceleration field $\bm{a}$ proportional
to this gradient:
\begin{equation}
  \bm{a}(\bm{x}) \propto -\,\nabla E_f(\bm{x}).
\end{equation}
The minus sign indicates flow towards regions of lower effective potential, in analogy with standard
potential theory, but here the potential is thermodynamic--structural rather than purely gravitational.

\subsection{D3. Expansion rate and background behaviour}

On large scales, an effective expansion rate $H$ can be linked to coarse-grained energy-flow variables.
A simple baseline relation uses the magnitude of $E_f$:
\begin{equation}
  H = H_0\, F(E_f, S),
\end{equation}
where $H_0$ is a reference scale and $F$ is a dimensionless function to be fixed by confrontation with
data (e.g.\ supernovae, BAO, CMB). The important point for this master specification is not the exact
form of $F$, but that $H$ is understood as a derived quantity from flow and entropy, not a primary
parameter.

\clearpage
\section{Part III --- EFC-C$_0$: Entropy--Cognition Base Layer}

\subsection{C$_0$. Entropy and information capacity}

EFC-C$_0$ links thermodynamic entropy to potential for information processing. The goal is not a
psychological model, but a base mapping between physical structure and abstract information capacity.

A local information capacity $I(\bm{x})$ is defined, at baseline, as
\begin{equation}
  I(\bm{x}) \propto \rho(\bm{x})\bigl(1 - S(\bm{x})\bigr).
\end{equation}
This mirrors the structure of $E_f$, but $I$ is interpreted as a potential for storing and transforming
information rather than driving motion directly.

\subsection{C$_1$. Local cognitive load}

For a coarse-grained region $\mathcal{R}$, define a total information capacity and a used fraction. The
total capacity is
\begin{equation}
  I_{\text{tot}}(\mathcal{R})
    = \int_{\mathcal{R}} I(\bm{x})\, dV.
\end{equation}
A simple scalar cognitive-load variable $L$ can then be defined as
\begin{equation}
  L = \frac{I_{\text{used}}}{I_{\text{tot}}}, \qquad 0 \leq L \leq 1,
\end{equation}
where $I_{\text{used}}$ is the part of the available capacity that is currently engaged in maintaining
or updating structure, patterns or internal models in the region.

\subsection{C$_2$. Informational field coupling}

EFC-C$_0$ treats information structures as coupled to the same energy-flow fields that drive dynamics
in EFC-D. At a coarse-grained level, one can express this by letting $I$ respond to changes in $E_f$:
\begin{equation}
  \frac{dI}{dt}
    = \alpha\,\frac{dE_f}{dt}
    - \beta\,D_I,
\end{equation}
where:
\begin{itemize}
  \item $\alpha$ scales how changes in energy-flow potential translate into increased or decreased
        information capacity,
  \item $\beta$ scales a dissipation term $D_I$ (for example diffusion, noise or degradation of
        structure).
\end{itemize}
This is a minimal base equation that later cognitive layers can extend.

\clearpage
\section{Appendix: Symbols and Definitions}

The table below summarises the main symbols used in this master specification.

\medskip

\begin{tabular}{lll}
  \hline
  Symbol & Meaning & Notes \\
  \hline
  $\rho$           & Mass density                 & $\rho = m/V$ \\
  $S$              & Dimensionless entropy        & Normalised to $[0,1]$ at chosen scale \\
  $E_f$            & Local energy-flow potential  & $E_f = \rho(1 - S)$ \\
  $\nabla_t$       & Time derivative              & Along chosen evolution parameter \\
  $\rho_h(r)$      & Halo mass density profile    & Part of EFC-S halo model \\
  $S_h(r)$         & Halo entropy profile         & Part of EFC-S halo model \\
  $I(\bm{x})$      & Local information capacity   & Base variable in EFC-C$_0$ \\
  $L$              & Cognitive load               & $L = I_{\text{used}} / I_{\text{tot}}$ \\
  $H$              & Effective expansion rate     & Derived from flow and entropy \\
  $H_0$            & Reference expansion scale    & To be calibrated against data \\
  $\alpha, \beta$  & Coupling coefficients        & Link between $E_f$ and $I$ \\
  \hline
\end{tabular}

\clearpage
\section*{How to Cite}

This document is the formal master specification for the Energy-Flow Cosmology (EFC) framework. It
defines the baseline mathematical and conceptual structure used across the EFC-S (Structure),
EFC-D (Dynamics) and EFC-C$_0$ (Entropy--Cognition) layers.

\medskip

\noindent \textbf{Citation:}\\
Magnusson, M. (2025). \textit{Energy-Flow Cosmology (EFC) --- Master Specification v1.1}. Figshare.\\
\href{https://doi.org/10.6084/m9.figshare.30630500}{DOI: 10.6084/m9.figshare.30630500}

\medskip

\noindent \textbf{DOI:}
\href{https://doi.org/10.6084/m9.figshare.30630500}{10.6084/m9.figshare.30630500}

\noindent \textbf{Author:} M. Magnusson \\
\textbf{ORCID:} \href{https://orcid.org/0009-0002-4860-5095}{0009-0002-4860-5095}

\medskip

\noindent \textbf{Licence:} CC-BY 4.0 (Creative Commons Attribution 4.0 International)

\medskip

\noindent This PDF corresponds to the archived and versioned research object stored on Figshare.
Future versions of the theory (EFC v1.x, v2.x) will reference this DOI as the baseline formal
specification.

\clearpage
\section{References}

This master specification is designed to be combined with an external reference list (articles,
datasets, code repositories). A fixed bibliography can be embedded here in later versions.

\end{document}
