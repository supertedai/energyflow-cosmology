\documentclass[11pt]{article}

\usepackage{graphicx}
\usepackage{hyperref}
\usepackage{geometry}
\usepackage{float}
\usepackage{amsmath}
\geometry{margin=2.5cm}

\title{EFC --- Entropy-Dependent Speed of Light Through the s₀/s₁ Phase Thresholds}
\author{Morten Magnusson}
\date{\today}

\begin{document}
\maketitle

\begin{abstract}
In the Energy-Flow Cosmology framework, the local value of the speed of light is a thermodynamic response variable determined by entropy-rate density. This short paper outlines how the phase transition between $s_0$ (zero-entropy state) and $s_1$ (max-entropy state) generates an effective acceleration of $c$ as a function of the underlying energy-flow gradient.
\end{abstract}

\section{Introduction}
In EFC, spacetime dynamics emerge from gradients in entropy and energy-flow. 
The value of $c$ is not placed into the system but arises as an emergent rate-limit on information propagation.

The $s_0/s_1$ boundary is of particular interest because it represents the extreme boundary conditions of the entropy spectrum. 
At $s_0$, entropy approaches zero, and at $s_1$, entropy saturates.

\section{Entropy Gradient and Phase Boundary}
The gradient between these two states forms a steep energy-flow slope. 
As this gradient increases toward the phase threshold, the effective propagation rate of information---what we measure as $c$---responds dynamically.

\section{Diagram}
\begin{figure}[H]
\centering
\includegraphics[width=\linewidth]{figures/s0s1-diagram.png}
\caption{Illustration of the s$_0$/s$_1$ phase boundary and its effect on the emergent speed of light.}
\end{figure}

\section{Conclusion}
The local speed of light is a thermodynamic product of entropy-rate density. 
The EFC formulation predicts that $c$ accelerates near the extreme entropy boundaries $s_0$ and $s_1$, offering a measurable signature of the underlying flow-based structure of spacetime.

\bibliographystyle{plain}
\bibliography{paper}

\end{document}
